% !TEX root = Projektdokumentation.tex

%Qualitätsmanagement: Messungen, Tests, Usability Tests, Code Review usw. (mit vollständiger Beschreibung der Anordnungen und Rahmenbedingungen)
\todo{Andrea}


\section{Usability}
Im Internet gibt es zahlreiche Online-Quizzes, auch für das schulische Umfeld. Damit Mobile Quiz häufig und gerne genutzt wird, gibt es einige Faktoren zu beachten. \cite{marketingfire.de} Dazu zählen unter anderem das Design und die Strukturierung der Seite. \\

Wie gut die bestehende Mobile Quiz - Version in diesen Bereichen abschneidet, kann mit einem \gls{Usability-Tests} festgestellt werden. Davon wurden zwei Durchführungen gemacht, wobei die erste zu Beginn der Arbeit dabei half, Schwierigkeiten in der Bedienung offenzulegen. Anschliessend flossen die Ergebnisse draus in die Aufgabenstellung mit ein. Gegen Ende der Arbeit fand dann die zweite Durchführung statt, um zu messen, welche Fortschritte durch die Arbeit gelungen sind.

\subsection{Methoden}
Bei den \gls{Usability-Tests} zu Beginn der Arbeit nahmen 3 Studenten der \acrfull{CN1}-Vorlesung sowie 1 Student aus dem 5. Semester teil, was der Zielgruppe von Mobile Quiz entspricht. Zudem hatten die Studenten aus \acrshort{CN1} erst wenig Erfahrung damit gesammelt. \\
Bei der Durchführung wurden die Teilnehmern in Situationen hineinversetzt, welche bei der Benutzung von Mobile Quiz oft vorkommen (siehe Usability-Test\_Aufgabenstellung). Die Teilnehmer wurden dabei 1 zu 1 beobachtet und Schwierigkeiten oder Abweichungen von den Erwartungen (s. Usability-Test\_Erwartungen) notiert. Die Gesamtauswertung wurde anschliessend in einem separaten Dokument festgehalten (s. Usability-Test\_Auswertung). Die erwähnten Dokumente befinden sich im Anhang.

\subsection{Erkenntnisse}
% Hier folgen sämtliche Erkenntnisse zum Bereich Usability.
% Auch die Auswertung was neu gemacht wird, kommt hier hin bzw. sicher eine Zusammenfassung und ganzes ist dann im Anhang.

\todo{Andrea}


\section{Codestatistik}

\todo{Andrea}


\section{Systemtests mit Selenium}
%automatisierte Systemtests mit Selenium
%Einleitung

\subsection{Methoden}
%Konkretes Vorgehen beschreiben

\subsection{Erkenntnisse}
%Was haben wir mit Hilfe von (siehe Titel) herausgefunden und wie werden wir es verbessern?

\section{Code Review}
%Code Review mit GitHub Branch
%Einleitung

\subsection{Methoden}
%Konkretes Vorgehen beschreiben

\subsection{Erkenntnisse}
%Was haben wir mit Hilfe von (siehe Titel) herausgefunden und wie werden wir es verbessern?

\section{Unit-Tests}
%Unit-Tests mit PHPUnit
%Einleitung

\subsection{Methoden}
%Konkretes Vorgehen beschreiben

\subsection{Erkenntnisse}
%Was haben wir mit Hilfe von (siehe Titel) herausgefunden und wie werden wir es verbessern?



