% !TEX root = Projektdokumentation.tex

\newglossaryentry{responsives CSS}{name={responsives CSS},description={CSS, welches die Seiteninhalte auf allen Bildschirmgrössen gut aussehen lässt.}}

\newglossaryentry{TLS}{name={TLS},description={Transport Layer Security ist ein Protokoll zur Sicherung der Kommunikation zwischen Client und Server.}}


\section{Eingesetzte Technologien und Werkzeuge}
MobileQuiz in der Version 3 verwendete die folgenden Technologien:

\begin{itemize}
	\item Apache2 als Web-Server
	\item PHP 5.6 als Server-seitige Programmiersprache
	\item MySQL als Datenbank
	\item HTML, CSS, JavaScript für den Seitenaufbau und die Interaktion
	\item jQuery für kürzere JavaScript-Befehle
	\item Bootstrap für \gls{responsives CSS}
\end{itemize}

Diese Technologiepalette wurde im Verlauf dieser Arbeit beibehalten und nicht erweitert. Die verwendeten Werkzeuge zur Entwicklung mit diesen Technologien sowie die restlichen, während dieser Arbeit eingesetzten Werkzeuge, sind im Anhang unter Kapitel \ref{chap:werkzeuge} aufgeführt. Beschrieben sind unter anderem der Einsatzzweck und der Bezugsort. Ergänzt wurden nützliche Hinweise, welche durch die Benutzung dieser Werkzeuge in Erfahrung gebracht wurden.


\section{Inbetriebnahme}
Das Aufsetzen der bestehenden Seite auf einer Server-Instanz der HSR dauerte länger als ursprünglich vorgesehen. Zu Beginn wurde das Aufsetzen mit Docker versucht, da die Server für die Studierenden neu damit ausgeliefert wurden. Anschliessend wurde eine neue Instanz mit reinem Linux Ubuntu bestellt, um die Seite ohne Docker in Betrieb zu nehmen. Beide Varianten scheiterten schlussendlich an der sicheren \gls{TLS}-Verbindung, wodurch diese dann im Code auskommentiert wurde.

Um ein weiteres Aufsetzen zu Erleichtern wurden zu beiden Varianten Anleitungen erstellt. Diese sind im Anhang unter \ref{sec:Anleitungen} zu finden.
Dort ist ebenfalls das Einrichten eines täglichen Backups für die Sicherung der Redmine-Datenbank beschreiben.


\section{Code-Änderungen}
Damit der Code auf der Server-Instanz der HSR lief, waren, neben dem Deaktivieren von \gls{TLS}, weitere Code-Änderungen nötig. Diese wurden sowohl von Patrick Eichler als auch von den Studenten durchgeführt und schriftlich festgehalten. Alle aufgetretenen Probleme und die dazugehörenden Lösungen sind im Anhang unter \ref{sec:AnfaenglicheCodeAenderungen} zu finden.
