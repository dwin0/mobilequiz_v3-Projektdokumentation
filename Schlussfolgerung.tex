% !TEX root = Projektdokumentation.tex

\section{Zusammenfassung}
Im Rahmen dieser Studienarbeit wurden die bereits bestehende Mobile Quiz Anwendung um neue Funktionalitäten erweitert und benutzerfreundlicher gestaltet. Dabei war ein grosser Teil der Veränderungen Überarbeitung der bestehenden Version und nur ein kleiner Anteil Ausarbeitung von neuen Funktionalitäten.

\bigskip

Die verbesserte Benutzerfreundlichkeit konnte erreicht werden, in dem die Inputs aus den \gls{Usability-Test}s und den eigenen Untersuchgen in die Erstellung des neuen Seitendesigns mit einflossen. Die Bestätigung, dass die vorgenommen Änderungen Wirkung zeigten, ergab sich aus den zum Schluss durchgeführten Usability-Tests. Die zusätzlichen Fehlerbehebungen zu Beginn der Studienarbeit erhöhten ausserdem die Zuverlässigkeit von Mobile Quiz.

\bigskip

Die neuen Funktionalitäten ergänzen die Online Quiz-Plattform in verschiedenen Bereichen. Mit den Bild-Fragen ergeben sich für den Ersteller eines Quizzes neue Möglichkeiten, das Wissen der Teilnehmer zu überprüfen. Auch diese profitieren von dieser Neuerung, denn das Lösen eines Quizzes wird damit spannender.
Mit der Möglichkeit, während der Beantwortung einer Frage, den Ersteller des Quizzes zu kontaktieren, sollen Unklarheiten möglichst schnell aus dem Weg geschafft werden.

\bigskip

Alles in allem konnte Mobile Quiz durch die Umsetzung der oben beschriebenen Punkte zu einer verlässlicheren und bedienungsfreundlicheren Quiz-Plattform umgestaltet werden.

\section{Ausblick}
Die neu Entwickelte Version wird ab dem nächsten Semester produktiv eingesetzt. Vor der effektiven Einsetzung müssen allerdings noch einige kleinere Arbeiten erledigt werden, für deren Umsetzung die Zeit zum Schluss der Studienarbeit leider nicht mehr reichte.

\bigskip

Zum einen muss die Implementation der Durchführung noch fertiggestellt werden, denn ohne diese können keine lauffähigen Quizzes erstellt werden. Zudem gibt es eine Liste mit vielen kleineren Aufgaben zur Weiterentwicklung, welche durch die HSR erledigt werden sollten. Darunter befinden sich Aufgaben wie:
\begin{itemize}
	\item Fehlermeldung bei zu grossen Bildern\\
	Wenn ein zu grosses Bild hochgeladen wird, wird dieses vom Server nicht angenommen. Zur Zeit wird dem Ersteller eine allgemeine Fehlermeldung angezeigt. Diese Fehlermeldung müsste so angepasst werden, dass der Ersteller das Problem auf den ersten Blick erkennt.
	\item Umsetzung der Verbesserungsvorschläge aus den zweiten Usability-Tests
\end{itemize}

\noindent Die vollständige Liste befindet sich im Kapitel \ref{subsec:NichtFertiggestellteArbeiten}.

\bigskip

Es wurden auch Konzepte ausgearbeitet, für welche in der Implementierungsphase keine Zeit mehr vorhanden war. Deshalb ist es gut möglich, dass es eine weitere Studienarbeit für Mobile Quiz geben kann. Darin könnten zum Beispiel die in der Analysephase ausgearbeiteten Statistiken umgesetzt werden. Es wurden zudem bereits einige Mockups erstellt, für welche die Zeit ebenfalls nicht mehr ausreichte.

\bigskip

\noindent Eine vollständige Liste aller Ideen für eine folgende Studienarbeit befindet sich im Kapitel \ref{sec:InhalteFuerStudentenarbeiten}.