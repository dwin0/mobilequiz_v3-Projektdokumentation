% !TEX root = Projektdokumentation.tex

\section{Zusammenfassung}
Im Rahmen dieser Studienarbeit wurden die bereits bestehende Mobile Quiz Anwendung um neue Funktionalitäten erweitert und Benutzerfreundlicher gemacht. Wobei ein grosser Teil der Veränderungen Überarbeitung des bestehenden Mobile Quizzes war und nur ein kleiner Anteil die Ausarbeitung von neuen Funktionalitäten.

\bigskip

Die verbesserte Benutzerfreundlichkeit konnte dadurch erreicht werden, in dem die Inputs aus den \gls{Usability-Test}s und den eigenen Untersuchgen in die Erstellung der neuen Seitendesigns mit einflossen. Die Bestätigung, dass die vorgenommen Änderungen zu einer benutzerfreundlicheren Bedienung führte, ergab sich mit den zum Schluss durchgeführten \gls{Usability-Test}s. Die zusätzliche Fehlerbehebungen zu Beginn der Studienarbeit erhöhte die Zuverlässigkeit von Mobile Quiz ebenfalls.

\bigskip

Die neuen Funktionalitäten ergänzen die Online Quiz-Plattform in verschiedenen Bereichen. Mit den Bild-Fragen ergeben sich für den Ersteller eines Quizzes neue Möglichkeiten, Fragen zu stellen. Auch der Teilnehmer profitiert von dieser Neuerung, denn das Lösen eines Quizzes wird für ihn spannender.
Mit der Möglichkeit, während dem Lösen eines Quizzes, den Ersteller des Quizzes zu kontaktieren, sollen Unklarheiten möglichst schnell aus dem Weg geschafft werden.

\bigskip

Alles in allem konnte Mobile Quiz durch die Umsetzung der oben beschriebenen Punkte zu einer verlässlicheren und bedienungsfreundlicheren Quiz-Plattform umgestaltet werden.

\section{Ausblick}
Die neu Entwickelte Version wird ab dem nächsten Semester produktiv eingesetzt. Vor der effektiven Einsetzung müssen allerdings noch einige kleinere Arbeiten erledigt werden, für die es zum Schluss der Studienarbeit leider nicht mehr reichte.

\bigskip

Zum einen muss die Implementation der Durchführung noch fertiggestellt werden, denn ohne die fertige Implementation, können keine gültigen Durchführungen erstellt werden. Zudem gibt es eine Liste mit vielen kleineren Aufgaben zur Weiterentwicklung, welche durch die HSR erledigt werden sollten. Darunter befinden sich Aufgaben wie:
\begin{itemize}
	\item Fehlermeldung bei zu grossen Bildern
	Wenn ein zu grosses Bild hochgeladen wird, wird dieses vom Server nicht angenommen. Zur Zeit wird dem Ersteller eine allgemeine Fehlermeldung angezeigt. Diese Fehlermeldung müsste so angepasst werden, dass der Ersteller das Problem auf den ersten Blick erkennt.
	\item Umsetzung der Verbesserungsvorschläge aus den zweiten \gls{Usability-Test}s
\end{itemize}

\noindent Die vollständige Liste befindet sich im Kapitel \ref{subsec:NichtFertiggestellteArbeiten}.

\bigskip

Es wurden auch Konzepte ausgearbeitet, für welche es in der Implementierungsphase keine Zeit mehr hatte. Deshalb ist es gut möglich, dass es eine weitere Studienarbeit für Mobile Quiz geben kann. Darin könnten dann zum Beispiel die in der in der Analysephase ausgearbeiteten Statistiken umgesetzt werden. Es wurden zudem bereits einige Mockups erstellt, für welche die Zeit nicht mehr reichte um diese umzusetzen.

\bigskip

\noindent Eine vollständige Liste aller Ideen für eine folgende Studienarbeit befindet sich im Kapitel \ref{sec:InhalteFuerStudentenarbeiten}.