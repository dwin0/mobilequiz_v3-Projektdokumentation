\documentclass[12pt, a4paper]{report}

\usepackage[utf8]{inputenc}
\usepackage[ngerman]{babel}
\usepackage{hyperref}
\usepackage{a4wide}
\usepackage{times}
\usepackage{pdfpages}
\usepackage{tabularx}
\usepackage[section]{glossaries}
\usepackage{todonotes}
\usepackage{multicol}
\usepackage{listings}
\usepackage{etoolbox}
\usepackage{subcaption}
\usepackage{fancyhdr}
\usepackage{xcolor}

%graphics
\usepackage{wrapfig}
\usepackage{graphicx}
\usepackage{float}

\usepackage{etoolbox}
\patchcmd{\chapter}{\thispagestyle{plain}}{\thispagestyle{fancy}}{}{}

\pagestyle{fancy}
\fancyhf{}
\renewcommand\headrulewidth{0pt}
\cfoot{\thepage\ von \pageref{LastPage}}
\fancyhead[LE, RO]{Studienarbeit: Mobile Quiz}
\makeglossaries
\setcounter{tocdepth}{1}

\fancypagestyle{mylandscape}{%
	\fancyhf{}% Clear header/footer
	\fancyfoot{% Footer
		\makebox[\textwidth][r]{% Right
			\rlap{\hspace{\footskip}% Push out of margin by \footskip
				\smash{% Remove vertical height
					\raisebox{\dimexpr.5\baselineskip+\footskip+.5\textheight}{% Raise vertically
						\rotatebox{90}{\thepage   von  \pageref{LastPage}}}}}}}% Rotate counter-clockwise
	\renewcommand{\headrulewidth}{0pt}% No header rule
	\renewcommand{\footrulewidth}{0pt}% No footer rule
}

\date{\today}
\title{Studienarbeit Mobile Quiz}
\author{Andrea Hauser und David Windler}



\begin{document}
	
	\begin{titlepage}
	\includepdf{PDFs/Titelblatt}
	\end{titlepage}
	
	\noindent{\LARGE \textbf{Abstract}}
	
	\bigskip
	
	% !TEX root = Projektdokumentation.tex

%Die Kurzzusammenfassung (Abstract) richtet sich an Leute, die den Themenkreis der Arbeit relativ gut kennen. Für diese Leute beschreiben Sie die neuen, eigenen Resultate der Arbeit. Die Kurzzusammenfassung soll nur etwa 200 Worte (etwa 20 Zeilen) lang sein. Bei Studienarbeiten ist das von der HSR Schulleitung vorgegebene Kurzzusammenfassungsformular zu verwenden.

Mobile Quiz ist eine Online-Quiz Plattform für Computer und Smartphones, welche in verschiedenen HSR-Modulen (Computernetze, Informations- und Codierungstheorie, Informationssicherheit) und Weiterbildungskursen regelmässig eingesetzt wird. Mobile Quiz entstand 2012 aus einer Bachelorarbeit und wurde seither mehrmals erweitert. Die zu Beginn der Arbeit vorliegende Mobile Quiz Version umfasste zwar viele praktische Funktionen und Einstellungs-möglichkeiten, es mangelte aber an der Bedienbarkeit. Im Rahmen dieser Studienarbeit sollten einerseits die Benutzerfreundlichkeit erhöht und andererseits neue Funktionen hinzugefügt werden.

\bigskip

In einem ersten Schritt wurde Mobile Quiz gründlich untersucht. Mit einem \gls{Usability-Test} wurden Optimierungsmöglichkeiten für Quizteilnehmer bestimmt. Anhand der Behebung von kleinen Fehlern machte man sich mit dem Code und den eingesetzten Technologien vertraut. Diese umfassen PHP, HTML, CSS, Javascript, JQuery und Bootstrap. Im Rahmen einer Umfeldanalyse wurden ähnliche Online Quiz-Plattformen gesucht, getestet und bewertet. Die aus dieser Projektphase gewonnenen Erkenntnisse halfen bei der Neugestaltung der Seiteninhalte sowie bei der Festlegung von neuen Funktionen. Beim Design wurden die angezeigten Informationen bewusst auf das Nötigste beschränkt. Zur Verbesserung der Benutzerführung wurden Symbole durch textuelle Menus ersetzt. Die Implementierung erfolgte während fünf Wochen. Abgeschlossen wurde die Arbeit mit einem Usability-Test.

\bigskip

Die Bedienbarkeit von Mobile Quiz wurde durch diese Studienarbeit sowohl für Quiz-Ersteller, als auch für Teilnehmer wesentlich verbessert. Die Schritt-für-Schritt Benutzerführung erleichtert die Erstellung von Quiz, Fragen und Durchführungen. Dank der neuen Excel-Import Funktion lassen sich Quiz und Fragen einfacher erstellen. Durch die Erweiterung \glqq Fragen mit Bildern\grqq sind attraktivere Fragestellungen möglich. Die Konzeptänderung, welche pro Quiz mehrere Durchführungen möglich macht, erleichtert den Einsatz von Mobile Quiz im Unterricht mit mehreren Übungsgruppen. Dank dem neuen Design sollten sich die Quizteilnehmer schneller zurechtfinden. Dies belegt der Vergleich der Ergebnisse der beiden Usability-Tests vor und nach der Überarbeitung des Mobile Quiz. Die jetzt vorliegende Mobile Quiz Version wird ab dem nächsten Semester produktiv eingesetzt. Die Umsetzung der in der Analysephase ausgearbeiteten statistischen Auswertungen könnte im Rahmen einer weiteren Studienarbeit erfolgen.
	\newpage
	
	
	\noindent{\LARGE \textbf{Aufgabenstellung}}
	\bigskip
	%Die unterschriebene Aufgabenstellung des Dozenten.
	% !TEX root = Projektdokumentation.tex

\noindent
{\renewcommand{\arraystretch}{1.5}
\begin{tabular}{l l}
	Studiengang: & Informatik (I) \\ 
	\hline
	Institut: & ITA: Internet-Techn. und Anwendungen \\ 
	Gruppe: & Andrea Hauser und David Windler \\ 
	\hline 
	Betreuer: & Prof. Dr. Peter Heinzmann (Dozent) und Patrick Eichler (Assistent)
\end{tabular} 
}

\bigskip\bigskip
\noindent{\large \textbf{Ausgangslage}}\\

In den Modulen Computernetze und Informationssicherheit können die Studierenden nach jeder Vorlesung ihr Wissen zum Vorlesungsstoff mit Hilfe der Webanwendung \url{www.mobilequiz.ch}  überprüfen. Die Studenten A. Hauser und D. Windler haben im Rahmen ihres Software Engineering Projekts ein Lernprogramm zum Thema AES Galois/Counter Mode erstellt. Sie wollten daher ihre Studienarbeit zum Thema Lernkontrollen durchführen.

Die MobileQuiz Lernkontrollen Anwendung wurde in den letzten Jahren im Rahmen von verschiedenen Studienarbeit entwickelt und erweitert. Seit der Überarbeitung durch HSR Assistent P. Eichler funktioniert \url{www.mobilequiz.ch} so stabil, dass es sogar in Prüfungen eingesetzt werden kann.  Dennoch gibt es Verbesserungs- und Erweiterungsmöglichkeiten, welche im Rahmen einer Studienarbeit bearbeitet werden können.

\bigskip\bigskip
\noindent{\large \textbf{Ziel}}\\

Bei der bestehenden Anwendung \url{www.mobilequiz.ch} zur Durchführung von Lernkontrollen sollen weitere Fragetypen unterstützt werden. Es sollen ausführliche statistische Auswertungen zur Qualität von Fragen und Antworten möglich sein. Ferner soll mit einer generellen Überarbeitung die Bedienfreundlichkeit verbessert werden.

\bigskip\bigskip
\noindent{\large \textbf{Aufgaben}}

\begin{enumerate}
	\item Einarbeitung, Analyse
	\begin{itemize}
		  \item Evaluation verschiedener Systeme zur Durchführung von Lernkontrollen (Vergleichstabelle)
		  \item Detaillierte Analyse der existierenden Anwendung \url{www.mobilequiz.ch} zur Durchführung von Lernkontrollen 
		  \item Durchführung von Usability Tests
		  \item Studium verschiedener theoretischer Untersuchungen zur computerbasierten Durchführung von Lernkontrollen (Abklärung und Übersicht zu Fragetypen)
		  \item Inbetriebnahme von Entwicklungs- und Dokumentationswerkzeugen
	\end{itemize}

	\item Refactoring
	\begin{itemize}
		  \item Inbetriebnahme von \url{www.mobilequiz.ch} auf HSR-Plattformen
		  \item Optimierung der aktuell vorhandenen Funktionen, Behebung von Fehlern (Offline Fragenerstellung und Excel Import/Export von Lernkontrollen, PDF Outputs, Sicherheitsprobleme)
	\end{itemize}

	\item Realisierung neuer Funktionen
	\begin{itemize}
		  \item Vereinfachung und Optimierung des Registrationsprozesses (Interessensgebiete, verbesserte Benutzerführung, Benachrichtigungen über neue Lernkontrollen), Erweiterung zur speziellen Behandlung von Benutzergruppen (Vorlesungsteilnehmende, Praktikumsgruppen)
		  \item Entwicklung neuer Fragetypen (Fragen mit Bildern)
		  \item Entwicklung neuer Antwortmöglichkeiten (Drag\&Drop)
		  \item Erweiterte statistische Auswertungen zu Fragen, Antworten und Resultaten der Teilnehmenden (Antwortzeit, Qualitätsmass für Fragen, Punktevergabe, Qualitätsmass für Antworten) 
		  \item Erweiterte Durchführungskontrolle (Zeitabstand von Wiederholungen bei mehrfachen Durchführungen)
	\end{itemize}
	
	\item Anpassungen für spezielle Use Cases
	\begin{itemize}
		\item Durchführung von Prüfungen
		\item Durchführung von Umfragen (Polls)
	\end{itemize}

	\item Dokumentation und Präsentation der Ergebnisse
  
\end{enumerate}

\bigskip\bigskip
{\large \textbf{Referenzen}}
\begin{enumerate}
	  \item Hinweise zur Durchführung von Studienarbeiten: \url{https://dl.dropboxusercontent.com/u/4679041/SABA-Web-Anleitungen\_Heinzmann.zip}
	  \item Aktuelle Anwendung \url{www.mobilquiz.ch}  
	  \item Manuela Grob, Quiz, HSR Bachelorarbeit, HS2010.
	  \item Patrik Naef, Khalid Abdul, «Mobile Quiz», HSR Bachelorarbeit, 16.9.2012.
\end{enumerate}

\bigskip \bigskip
\parbox{6cm}{Rapperswil, \hrule
\strut \footnotesize Ort, Datum} \hfill
\parbox{5cm}{\hrule
\strut \footnotesize Betreuer}

	\newpage
	
	
	\includepdf[pagecommand={}]{PDFs/Eigenstaendigkeitserklaerung.pdf}
	%Unterschriebenes Formular "Erklärung zur Urheberschaft" Handelt es sich um eine Fortsetzung einer Studienarbeit, so ist in der HSR_Erklaerung_Urheberschaft klar aufzuzeigen, was im Rahmen der Studienarbeit und was bei der Bachelorarbeit gemacht wurde. In diesem Formular ist auch anzugeben, welche Informationen Copyright geschützt sind und daher nicht ohne Weiteres weitergegeben werden können.
	\newpage
	
	
	\includepdf[pagecommand={}]{PDFs/Vereinbarungen_ueber_Urheber-und_Nutzungsrechte.pdf}
	%Unterschriebenes Formular "Vereinbarung zur Verwendung und Weiterentwicklung der Arbeit".
	\newpage
	
	
	\noindent{\LARGE \textbf{Management Summary}}
	
	\bigskip
	
	% !TEX root = Projektdokumentation.tex

Mobile Quiz ist eine Online-Quiz Plattform für Computer und Smartphones, welche in verschiedenen HSR-Modulen (Computernetze, Informations- und Codierungstheorie, Informationssicherheit) und Weiterbildungskursen regelmässig eingesetzt wird. Mobile Quiz entstand 2012 aus einer Bachelorarbeit und wurde seither mehrmals erweitert. Die zu Beginn der Arbeit vorliegende Mobile Quiz Version umfasste zwar viele praktische Funktionen und Einstellungs-möglichkeiten, es mangelte aber an der Bedienbarkeit.

\bigskip

Im Rahmen dieser Studienarbeit sollten einerseits die Benutzerfreundlichkeit erhöht und andererseits neue Funktionen hinzugefügt werden.

\bigskip

Mobile Quiz wird vor allem von Dozenten verwendet, um für Ihre Studenten Quizzes in Form von Lernhilfe während des Semesters zu bieten. Es besteht weiter die Möglichkeit die Quizzes entweder mit dem Typ Testatbedingung oder als Typ Prüfung zu erstellen und durchzuspielen. Die Wichtigsten Interaktionen mit dem Mobile Quiz System wurden in der folgenden Grafik festgehalten.
\begin{figure}[H]
	\centering
	\includegraphics[width=0.75\textwidth]
	{Images/MobileQuiz_Uebersicht.PNG}
	\caption{Übersicht der wichtigsten Funktionen des Mobile Quiz}
\end{figure}

\bigskip

Das Vorgehen um sich in das Thema einzuarbeiten wurde wie folgt gewählt. In einem ersten Schritt wurde Mobile Quiz gründlich untersucht. Mit einem \gls{Usability-Test} wurden Optimierungsmöglichkeiten für Quizteilnehmer bestimmt. Anhand der Behebung von kleinen Fehlern machte man sich mit dem Code und den eingesetzten Technologien vertraut. Im Rahmen einer Umfeldanalyse wurden ähnliche Online Quiz-Plattformen gesucht, getestet und bewertet. Die aus dieser Projektphase gewonnenen Erkenntnisse halfen bei der Neugestaltung der Seiteninhalte sowie bei der Festlegung von neuen Funktionen. Beim Design wurden die angezeigten Informationen bewusst auf das Nötigste beschränkt. Zur Verbesserung der Benutzerführung wurden Symbole durch textuelle Menus ersetzt. Die Implementierung erfolgte während fünf Wochen. Da es sich bei dieser Arbeit um eine Erweiterung eines bestehenden Systems handelt, wurden die Technologien nicht neu gewählt. Es wurde die bereits bestehende Lösung mit PHP und JQuery für die Logik, der MySQL Datenbank für das Speichern der Daten und HTML zusammen mit CSS für die Darstellung verwendet. Nach der Implementierungsphase wurde die Arbeit mit einem \gls{Usability-Test} abgeschlossen.

\bigskip

Als Ergebnis dieser Arbeit konnte eine sowohl für den Quiz-Ersteller wie auch für den Teilnehmer wesentlich verbesserte Bedienbarkeit erreicht werden. Die Schritt-für-Schritt Benutzerführung erleichtert die Erstellung von Quiz, Fragen und Durchführungen. Dank der neuen Excel-Import Funktion lassen sich Quiz und Fragen einfacher erstellen. Durch die Erweiterung \glqq Fragen mit Bildern\grqq sind attraktivere Fragestellungen möglich. Die Konzeptänderung, welche pro Quiz mehrere Durchführungen möglich macht, erleichtert den Einsatz von Mobile Quiz im Unterricht mit mehreren Übungsgruppen. Dank dem neuen Design sollten sich die Quizteilnehmer schneller zurechtfinden. Dies belegt der Vergleich der Ergebnisse der beiden \gls{Usability-Test}s vor und nach der Überarbeitung des Mobile Quiz. Die jetzt vorliegende Mobile Quiz Version wird ab dem nächsten Semester produktiv eingesetzt. Die Umsetzung der in der Analysephase ausgearbeiteten statistischen Auswertungen könnte im Rahmen einer weiteren Studienarbeit erfolgen.
	%Das "Management Summary" bzw. die Zusammenfassung richtet sich in der Praxis an die "Chefs des Chefs", d.h. an die Vorgesetzten des Auftraggebers (diese sind weniger tief in die Thematik involviert als die direkt Beteiligten). Das Management Summary soll maximal 4 Seiten umfassen und mindestens eine Figur enthalten. Die Sprache soll knapp und klar sein.
	
	%Im Management Summary braucht es keine Untertitel. Die jetzt erstellten Titel sind eher als Vorgabe für den Ablauf gedacht.
	
	
	%Ausgangslage
	%	·         Ausgangslage (Warum wurde das Projekt durchgeführt? Was machen andere und welche ähnlichen Arbeiten gibt es zum Thema? Welche Ziele wurden gesteckt (Muss-, Soll-, Nice-to-Have Ziele)
	
	
	%Vorgehen
	%   ·         Vorgehen (Was wurde gemacht? In welchen Teilschritten? Wer war involviert (Durchführung, Entscheide, Zwischenprüfungen/Feedbacks usw.)? Verwendete Werkzeuge)
	
	
	%Ergebnisse
	%   ·         Ergebnisse (Was ist das Resultat des Projekts (quantifizierbarer und qualitativer Nutzen)? Was musste man selbst tun, was konnte von anderen verwendet werden? Selbstbeurteilung der  Zielerreichung in Bezug auf die Zielsetzungen der Arbeit. Abweichungen von den Zielsetzungen und Begründung dafür (positiv und negativ). Kosten. Lernpunkte aus der Durchführung des Projekts)
	
	
	%Ausblick
	%   ·         Ausblick (Verbleibende Probleme, Risiken und Gegenmassnahmen, was würde man anders machen? Wie geht es mit dem Projekt weiter, wer nutzt was? Welche weiteren Schritte, Entwicklungen, Anpassungen etc. werden empfohlen / sind geplant?)
	
		
	
	\tableofcontents
	%Im Inhaltsverzeichnis sollen Sie nur die Stufen Kapitel (h.), Unterkapitel (h.i) bzw. im Anhang die Haupt- (X) und Unterabschnitte (X.n) aufführen.
	\newpage
	
	
	
	\part{Hauptbericht}
	
	\chapter{Einleitung}
	% !TEX root = Projektdokumentation.tex

% 	Die Einleitung soll allgemein verständlich sein, d.h. sie soll beispielsweise auch für Ihre Freunde und Verwandten verständlich sein. Sie stellt die Aufgabe in einen grösseren Zusammenhang und liefert eine genaue Beschreibung der Ausgangslage und Problemstellung. Allfällige Vorarbeiten oder ähnlich gelagerte Arbeiten sind diskutiert. Zum Schluss der Einleitung können Sie auch beschreiben, welche Abschnitte des Berichts sich an welche Leser wenden (z.B. Anwender des Produkts, Entwickler, Betreiber).


%Überprüfen von Wissen \\
Während dem Studium werden viele Inhalte vermittelt und anschliessend mit einer Schlussprüfung abgeholt. Wie merkt ein Student aber schon vor der Prüfung, ob sein Wissen sattelfest ist? Mobile Quiz bietet eine Lösung dafür. Der Dozent publiziert nach jeder Lektion Fragen zu den vermittelten Inhalten, welche die Studierenden online beantworten können. So erhalten Sie umgehend Feedback zu ihrem Wissensstand.
Die folgende Figur zeigt die wichtigsten Interaktionen der unterschiedlichen Benutzer mit dem Mobile Quiz:

\begin{figure}[H]
	\centering
	\includegraphics[width=0.75\textwidth]
	{Images/MobileQuiz_Uebersicht.PNG}
	\caption{Übersicht der wichtigsten Funktionen des Mobile Quiz}
\end{figure}

%Stand Mobile Quiz \\
Stand vor der Studienarbeit umfasst Mobile Quiz inzwischen einige Funktionen, um Quizzes zu erstellen. Verbesserungspotentiale liegen allerdings noch in den Bereichen der Bedienbarkeit, im Bereich von neuen Fragetypen sowie in der Auswertung von Quizzes. Durch letzteres wäre es dem Dozenten ersichtlich, welche Teile des Stoffs gut verstanden wurden und bei welchen noch Nachholbedarf herrscht. Somit könnte die Vorlesungszeit effizienter genutzt werden.

Die folgenden Bilder sollen die wichtigsten Ansichten in der alten Version des Mobile Quiz anzeigen. Die alte Version des Mobile Quiz ist unter \url{https://tlng.cnlab.ch/mobilequiz_v3} auffindbar.
\begin{figure}[H]
	\centering
	\includegraphics[width=0.75\textwidth]
	{Images/MobileQuizAlteVersionStartseiteErsteller.png}
	\caption{Ansicht der alten Mobile Quiz Startseite}
	\cite{mobilequiz.ch}
\end{figure}

\begin{figure}[H]
	\centering
	\includegraphics[width=0.75\textwidth]
	{Images/MobileQuizAlteVersionFrageErstellen.png}
	\caption{Ansicht der alten Mobile Quiz Fragen-Erstellungsseite}
	\cite{mobilequiz.ch}
\end{figure}

\begin{figure}[H]
	\centering
	\includegraphics[width=0.75\textwidth]
	{Images/MobileQuizAlteVersionQuizErstellen.png}
	\caption{Ansicht der alten Mobile Quiz Quiz-Erstellungsseite}
	\cite{mobilequiz.ch}
\end{figure}

\begin{figure}[H]
	\centering
	\includegraphics[width=0.75\textwidth]
	{Images/MobileQuizAlteVersionQuizDurchfuehrung.png}
	\caption{Ansicht der alten Mobile Quiz Quiz-Durchführungsseite}
	\cite{mobilequiz.ch}
\end{figure}

%In dieser Arbeit wird XY umgesetzt.
In dieser Arbeit die bessere Bedienbarkeit für Ersteller und Teilnehmer umgesetzt. Es wird dabei immer auch geschaut, dass die Mobile Version der Website bedienbar ist. Zudem werden neue Funktionalitäten, wie zum Beispiel das neue Excel für den Fragen-Import oder auch ein neuer Fragetyp implementiert.

\bigskip

%Übersicht Kapitel \\
Im folgenden Kapitel wird das bestehende Produkt genauer beschrieben und es wird darauf eingegangen, wie die Inbetriebnahme voranging. Das Kapitel drei befasst sich mit den zu Beginn vorgenommen Analysen im Bereich ähnliche Arbeiten, den eigenen Tests mit Mobile Quiz und der Umfeldanalyse. Im darauf folgenden Kapitel sind die erarbeiteten Konzepte zum Gruppenmanagement, den neuen Fragetypen und den Statistiken und Auswertungen zu finden. Das Kapitel Design zeigt die wichtigsten Überlegungen zu den Seitenneugestaltungen auf. Im darauf folgenden Kapitel Software Engineering werden die vorgenommen Datenbankänderungen beschrieben. Das Kapitel sieben befasst sich dann mit der Beschreibung der Implementation. Im Kapitel acht werden die Folgearbeiten beschrieben. Das nächste Kapitel setzt sich mit dem Qualitätsmanagement auseinander. Abgeschlossen wird der Hauptteil mit dem Kapitel zehn Schlussfolgerung.
	
	%Die nach der Einleitung folgenden Hauptabschnitte richten sich in der Regel an die Nutzer und Betreiber des realisierten Systems und an die im entsprechenden Fachgebiet tätigen Ingenieure. Die Hauptabschnitte bilden typischerweise die wichtigsten Ergebnisse Ihrer Arbeit ab. Sie beschreiben das realisierte System bzw. die getätigten Untersuchungen. Die Leser sollen die zur Problemlösung getätigten Überlegungen verstehen. Theoretische Grundlagen sind soweit aufzuführen, als dies für die Lösung der Aufgabe nötig ist (keine Lehrbücher oder Wikipedia-Artikel (ab)schreiben). Die Erkenntnisse aus den theoretischen Untersuchungen sind also wenn immer möglich direkt mit der Problemlösung zu verknüpfen (z.B. mit eigenen Messungen oder mit Beispielen aus der schliesslich erstellten Anwendung zu illustrieren).
	
	%Sparen Sie nicht mit Diagrammen und Figuren. Diese müssen auch im Text diskutiert sein, d.h. es wird auch in Worten beschrieben, was man mit dem Diagramm oder der Figur zeigen will. Spezielle Details, welche die Kontinuität in den Hauptabschnitten stören, sind im Anhang aufzuführen.
	
	%Der Umfang der einzelnen Abschnitte des Berichts entspricht in der Regel dem Arbeitsaufwand, welcher für die entsprechenden Themen eingesetzt wurde.
	
	%Je nach Aufgabenstellung können beispielsweise Hauptabschnitte der folgenden Art vorkommen:
	
	\chapter{Bestehendes Produkt / Inbetriebnahme}
	% !TEX root = Projektdokumentation.tex

\newglossaryentry{responsives CSS}{name={responsives CSS},description={CSS, welches die Seiteninhalte auf allen Bildschirmgrössen gut aussehen lässt.}}

\newglossaryentry{TLS}{name={TLS},description={Transport Layer Security ist ein Protokoll zur Sicherung der Kommunikation zwischen Client und Server.}}


\section{Eingesetzte Technologien und Werkzeuge}
MobileQuiz in der Version 3 verwendete die folgenden Technologien:

\begin{itemize}
	\item Apache2 als Web-Server
	\item PHP 5.6 als Server-seitige Programmiersprache
	\item MySQL als Datenbank
	\item HTML, CSS, JavaScript für den Seitenaufbau und die Interaktion
	\item jQuery für kürzere JavaScript-Befehle
	\item Bootstrap für \gls{responsives CSS}
\end{itemize}

Diese Technologiepalette wurde im Verlauf dieser Arbeit beibehalten und nicht erweitert. Die verwendeten Werkzeuge zur Entwicklung mit diesen Technologien sowie die restlichen, während dieser Arbeit eingesetzten Werkzeuge, sind im Anhang unter Kapitel ... aufgeführt. Beschrieben sind unter anderem der Einsatzzweck und der Bezugsort. Ergänzt wurden nützliche Hinweise, welche durch die Benutzung dieser Werkzeuge in Erfahrung gebracht wurden. \todo{Verweis auf Anhang}


\section{Inbetriebnahme}
Das Aufsetzen der bestehenden Seite auf einer Server-Instanz der HSR dauerte länger als ursprünglich vorgesehen. Zu Beginn wurde das Aufsetzen mit Docker probiert, da die Server für die Studierenden neu damit ausgeliefert wurden. Anschliessend wurde eine neue Instanz mit reinem Linux Ubuntu bestellt, um die Seite ohne Docker in Betrieb zu nehmen. Beide Varianten scheiterten schlussendlich an der sicheren \gls{TLS}-Verbindung, wodurch diese dann im Code auskommentiert wurde.

Um ein weiteres Aufsetzen zu Erleichtern wurden zu beiden Varianten Anleitungen erstellt. Diese sind im Anhang unter \glqq V2 Apache\_PHP\_MySQL-Docker.tex\grqq und \glqq V3 Apache\_PHP\_MySQL-Ubuntu.tex \grqq zu finden.
Unter \glqq Anleitung Redmine Datenbank Backup\grqq ist dort ebenfalls das Einrichten eines täglichen Backups für die Sicherung der MySQL-Datenbank beschreiben.
\todo{in Anhang einfügen und Verweis erstellen}


\section{Code-Änderungen}
Damit der Code auf der Server-Instanz der HSR lief, waren, neben dem Deaktivieren von TLS, weitere Code-Änderungen nötig. Diese wurden sowohl von Patrick Eichler als auch von den Studenten durchgeführt und schriftlich festgehalten. Alle aufgetretenen Probleme und die dazugehörenden Lösungen sind im Anhang unter \glqq Anfängliche\_Code\_Änderungen.pdf\grqq zu finden.
\todo{Code-Änderungen schön niederschreiben, in den Anhang und Verweis darauf}
\todo{Änderungen rückgängig machen? wenn ja, nicht alle, da manche sinnvoll}










	
	\chapter{Analyse}
	% !TEX root = Projektdokumentation.tex


\newglossaryentry{Cross-Site-Scripting}{name={Cross-Site-Scripting},description={
		Cross-site scripting (XSS) ist ein Angriff auf eine Webapplikation, die Benutzereingaben nicht sorgfältig überprüft, bevor diese wieder an weitere Benutzer zurückgeschickt werden. So kann ein Angreifer ausführbaren Code mitgeben, der anschliessend bei vielen anderen Benutzern im Browser ausgeführt wird \cite{whatIs_xss}}}

\newglossaryentry{Vulnerability}{name={Vulnerability},description={
		Fehler im Code oder Design, welcher ein potentielles Sicherheitsrisiko darstellt \cite{whatIs_vulnerability}}}

\newacronym{IKF}{IKF}{Institut für Kommunikation \& Führung}

\newacronym{SKMF}{SKMF}{Swiss Knowledge Management Forum}

%Analyse der Aufgabenstellung, Requirements Engineering, Umfeldanalyse (vergleichbare Produkte und Lösungen), Resultate der Literaturrecherche
%Auch: Ziel und Zweck von Mobile Quiz, wofür und vom wem wird es eingesetzt? Was ist zukünftiges Ziel?

% Hier werden alle Erkenntnisse aus den einzelnen Unterkapitel zusammengefasst.
%ohne Titel

%Hier noch eine Zusammenfassung
Um ein möglichst gutes Bild davon zu erhalten, was im Bereich Online-Quizzes bereits vorhanden ist und wo Mobile Quiz aktuell steht, wurden Informationen in verschiedenen Bereichen gesucht und zusammengetragen.

Dazu wurden unter anderem ähnliche Arbeiten gesucht (Punkt 8.2) und diese auf ihre Relevanz überprüft. Dabei wurde festgestellt, dass sich vor allem die HSR auf Quizzes im Lernbereich konzentriert.
Das Testen von Mobile Quiz selbst (Punkt 8.3.1) zeigte, dass es noch einige Probleme in der Version 3 anzutreffen gab. Diese zu beheben würde die Plattform solider machen und bestenfalls neue Quiz-Ersteller anziehen.
Weiter wurde beim Vergleich mit anderen Online-Quiz-Plattformen (Punkt 8.3.2) ersichtlich, dass Mobile Quiz im Bereich Funktionsumfang und Einstellungsmöglichkeiten gut dastand. Allerdings wirkte sich die Usability schlecht darauf aus, weil viele Funktionen nicht sofort ersichtlich oder nur schwierig zu erreichen waren. Dies wurde auch durch die Usability-Tests (Abschnitt 12.1) bestätigt, welche ebenfalls zu Beginn der Arbeit durchgeführt wurden.

Die Ergebnisse aus dieser Untersuchung flossen zusammen mit den bereits bekannten Verbesserungspunkten in die mögliche Aufgabenstellung dieser Studienarbeit ein. Diese ist im Anhang unter 'MoeglicheArbeitenSA.pdf' ersichtlich. Um nicht jeden Bug einzeln aufzuführen, wurden darin die Probleme abstrahiert und nur Themenbereiche aufgeführt und beschrieben. Diese wurden dann mit dem Betreuer besprochen und priorisiert, um daraus die definitive Aufgabenstellung zu erstellen.

\newpage

\section{Recherche/ähnliche Arbeiten}

Welche ähnlichen Arbeiten, seien es Bachelorarbeiten, Studienarbeiten oder sonstige Arbeiten in ähnlichem Umfang, gibt es bereits?

Da es kein zentrales, Schulen übergreifendes Verzeichnis aller Arbeiten gibt, musste zur Beantwortung dieser Frage die einzelnen Verzeichnisse der Schulen durchgegangen werden. Dabei wurden die folgende Schulen berücksichtigt:
\begin{itemize}
	\item ETH Zürich
	\item ZHAW
	\item Universität Zürich
	\item HSR
\end{itemize}

\bigskip

Welche zu folgendem Ergebnis führte:
\begin{itemize}
	\item ETH Zürich
	\begin{itemize}
		\item Die Arbeiten, welche an der ETH Zürich erstellt wurden, behandeln spezifisch auf die Prüfungssituation ausgelegte Tools. Diese Arbeiten wurden deshalb nicht weiter im Detail betrachtet. 
	\end{itemize}
	\item ZHAW
	\begin{itemize}
		\item Im Verzeichnis aller Arbeiten stach die Arbeit "Edu4u. Geschäftsmodell einer Webplattform im E-Learning-Bereich für E-Lectures, Online-Kurse und Filmdokumentationen" heraus. Leider konnte diese Arbeit nicht genauer angeschaut werden, da sie als vertraulich klassifiziert wurde.
	\end{itemize}
	\item Universität Zürich
	\begin{itemize}
		\item Hier wurde leider kein Verzeichnis der Arbeiten gefunden.
	\end{itemize}
	\item HSR
	\begin{itemize}
		\item Studienarbeiten Crowdsourced Quizzes und Crowdsourced Quizzes 2, Quizzenger als Ergebnis.
		\item Bachelorarbeit MobileQuiz, Vorgängerarbeit unserer Studienarbeit.
		\item Bachelorarbeit Digital Native Quiz
	\end{itemize}
\end{itemize}

Bei der Suche nach ähnlichen Arbeiten, wurde auch der Standard \glqq IMS Question \& Test Interoperability\grqq \cite{imsglobal.org} der IMS Global Learning Consortium entdeckt. Dieser Standard beinhaltet eine umfassende Übersicht von möglichen Fragetypen.

Das IMS Global Learning Consortium setzt sich allgemein dafür ein, dass ein gemeinsamer Standard entsteht, um für Interoperabilität zwischen einzelnen Lernseiten zu sorgen. Bei vertieften Abklärungen wurde festgestellt, dass sowohl Moodle als auch TAO (zwei der Open Source Seiten, welche in der Webuntersuchung angeschaut wurden) sich auf die Standards von IMS Global Learning Consortium ausrichten.

Wir haben uns allerdings dagegen entschieden, auf die Standards von IMS Global zu wechseln, da es sich um einen zu grossen Aufwand handeln würde. Und weil sich das MobileQuiz nicht auf absehbarer Zeit mit anderen Seiten austauschen wird.


\section{Eigene Untersuchungen, Webuntersuchung}

Was kann Mobile Quiz heute bereits, wo gibt es noch Probleme und wo steht das Quiz heute im Vergleich zu ähnlichen Webanwendungen? Diese Fragen waren Kern der eigenen Untersuchungen. Einerseits wurde dazu Mobile Quiz selbst intensiv getestet, andererseits wurden mehrere vergleichbare Online-Quizzes gesucht und anhand von bestimmten Kriterien verglichen.


	\subsection{Untersuchung www.mobilequiz.ch}
	Während mehrerer Stunden wurden mit dem eigenen Benutzeraccount die vorhandenen Funktionen ausprobiert. Dabei kamen verschiedene Probleme zutage, die sich grob in die folgenden Kategorien unterteilen lassen:
	
	
	\begin{itemize}
		\item Sicherheitsrelevante Probleme \\
		\textit{Beispiel}: Wird eine Frage erfasst, so kann im Fragetext mittels HTML-Script-Tag JavaScript hinterlegt werden, welches beim Anzeigen der Frage beim Teilnehmer ausgeführt wird. Somit hat Mobile Quiz gegenüber Frage-Erstellern eine \gls{Cross-Site-Scripting} - \gls{Vulnerability}. Diese könnte ausgenutzt werden, um das Session-Cookie eines Benutzers zu stehlen.
		\item Usability \\
		\textit{Beispiel}: In der Lernkontrollen-Übersicht ist der Start-Button zu wenig ersichtlich.
		\item Probleme/Bugs \\
		\textit{Beispiel}: Wird bei einer Lernkontrolle festgelegt, dass sie keinen Endzeitpunkt hat, so soll auch kein Enddatum angezeigt werden.
		\item Schreibfehler/Grammatik \\
		\textit{Beispiel}: Der Benutzer wird wird teilweise mit 'Sie' angesprochen, andernorts wird die 'Du'-Form verwendet.
		\item Allgemeine Fragen zum Konzept \\
		\textit{Beispiel}: Muss bei einer Registrierung wirklich die Adresse angegeben werden?
		\item Mobile-Probleme \\
		\textit{Beispiel}: Der Zugriff via Smartphone auf den Profilbereich funktioniert nicht.
	\end{itemize}

	Die ausführlichen Resultate sind im Dokument 'Ergebnisse eigene Tests.pdf' ersichtlich.
	

	\subsection{Vergleich Online-Quizzes}
	Um andere Online-Quizzes mit Mobile Quiz zu vergleichen mussten zuerst die Kriterien festgelegt werden. Als Grundlage diente eine Excel-Tabelle von Khalid Abdul und Patrik Naef, welche an die Bedürfnisse dieser Arbeit angepasst wurde.
	Die Vergleichs-Systeme wurden durch Google-Suchen und Empfehlungen von educatorstechnology.com \cite{educatorstechnology.com} ausgewählt.
		
	Um sicher zu sein, dass die durch uns gewählten Webseiten einen repräsentativen Anteil der Lernquizzes abdeckt, haben wir uns mit Institutionen und Personen in Verbindung gesetzt, welche in diesem Umfeld tätig sind. Um Unterstützung angefragt haben wir bei Switch AAA, dem \acrfull{IKF} und beim \acrfull{SKMF}.
	Leider hatte bei Switch AAA niemand Zeit für uns. Beim \acrshort{IKF} haben Sie selbst keine solche Vergleiche, oder falls etwas vorhanden ist, handelt es sich um Unterrichtsmaterial, welches nicht herausgegeben wird. Das \acrshort{SKMF} konnte nur via ein Webformular kontaktiert werden und hat sich bis am 04.11.2016 noch nicht zurückgemeldet.
	
	\bigskip
	
	Die detaillierte Auswertung ist im Dokument 'SA-Mobile-Quiz\_QuizSysteme-Funktionen-Vergleich-Matrix.xlsx' ersichtlich. Nachfolgend werden wichtige Punkte daraus genauer aufgezeigt:
	
	\begin{itemize}
		\item Willkommensseite \\
		Webanwender haben eine riesige Auswahl an Seiten, welche auf ihr Bedürfnis zugeschnitten sind. Sie wenden deshalb nicht viel Zeit auf, um sich über eine einzelne Seite genauer zu informieren. Aus diesem Grund zählt oft der erste Eindruck, also eine ansprechende Willkommensseite. \\
		
		\begin{figure}[H]
			\centering
			\includegraphics[width=0.75\textwidth]{Images/MobileQuiz_StartPage.PNG}
			\caption{Startseite Mobile Quiz Version 3}
		\end{figure}
		
		Ruft man www.mobilequiz.ch auf, so sieht man viel Text, der die Seite beschreibt. Ein Benutzer weiss jedoch noch nicht genau, was ihn erwartet, wenn er sich registriert und einloggt.
		
		\begin{figure}[H]
			\centering
			\includegraphics[width=0.75\textwidth]
			{Images/Qzzr_StartPage_Statistics.PNG}
			\caption{Startseite Qzzr}
		\end{figure}
				
		Seiten wie Qzzr \cite{qzzr.com} hingegen, zeigen anhand von Bildern und Symbolen auf, was die Funktionalitäten sind und wie diese konkret aussehen. Solche Bilder sind schnell erfasst  und verarbeitet.
		Mobile Quiz könnte die gleichen Funktionsumfang bieten, aber wenn es der Benutzer nicht sofort sieht, klickt er weiter und registriert sich andernorts.
		
		Was machen gute Willkommensseiten also aus?
		
		\todo{David: Internetrecherche}
		http://blog.hubspot.com/blog/tabid/6307/bid/34006/15-examples-of-brilliant-homepage-design.aspx
		
		
		
		\item Schritt für Schritt - Erstellung von Quizzes \\
		Um ein Quiz zu erstellen benötigt es einerseits die Fragen, andererseits das Quiz selbst, welches mehrere Fragen umfasst. In welcher Reihenfolge sollen diese beiden Ressourcen erstellt werden? \\
		Bei Mobile Quiz war der Ablauf so geregelt, dass zuerst die Fragen und anschliessend das Quiz separat erstellt wurde. War man sich dieser Tatsache bewusst, so stellte dies kein Problem dar, aber war es auch intuitiv? Wie in den durchgeführten Usability-Tests festgestellt wurde, war dem nicht so. Die Benutzer starteten sofort mit der Erstellung des Quizzes, mussten dann aber abbrechen, weil darin keine neuen Fragen erfasst werden konnten.
		Aus diesem Grund war es sinnvoll, diese Reihenfolge in Mobile Quiz zu ändern. Hier bot Testmoz \cite{testmoz.com} ein gutes Vorgehen:
		
		\begin{enumerate}
			\item Testnamen eingeben
			\item Testeinstellungen vornehmen
			\item Fragen erfassen
			\item Veröffentlichen
			\item Reports anschauen
		\end{enumerate}
		
		\begin{figure}[H]
			\centering
			\includegraphics[width=0.4\textwidth]{Images/Testmoz2.PNG}
			\caption{Ablauf Testmoz}
		\end{figure}
		
		Der Ablauf von Mobile Quiz war einzig darin zu ändern, dass neue Fragen während der Erstellung eines Quizzes erfasst werden können. Zur Vereinfachung konnte auch beitragen, dass der Ablauf wie bei Testmoz \cite{testmoz.com} verteilter dargestellt wird, sodass pro Seite weniger Informationen stehen. Somit findet sich der Benutzer schneller zurecht.
		
		
		
		
		\item  Quiz-Einstellungen \\
		Quizzes können für unterschiedliche Bedürfnisse eingesetzt werden. Die möglichen Einsatzzwecke reichen von Freunden, die zum Zeitvertreib ihr Wissen gegenseitig messen wollen, über Dozenten, die prüfen möchten, ob die Studenten den Unterrichtsstoff verstanden haben, bis zu Dozenten, welche die Quizzes als Prüfung verwenden.
		Diese Situationen verlangen viele Einstellungsmöglichkeiten, welche für den Benutzer möglichst selbsterklärend sein sollen. Trifft dies jedoch nicht zu, oder ist die Darstellung unverständlich, so wird sich der Quiz-Ersteller möglicherweise nach einer anderen Quiz-Plattform umsehen.
		
		\begin{figure}[H]
			\centering
			\includegraphics[width=0.75\textwidth
			]{Images/MobileQuiz_Quiz-Settings.PNG}
			\caption{Quiz-Einstellungen Mobile Quiz Version 3}
		\end{figure}
		
		
		Mobile Quiz bot zwar viele Einstellungsmöglichkeiten an, diese waren jedoch so zahlreich, dass sie den Benutzer fast überforderten. Zudem war die Darstellung zum Teil nicht optimal, da beispielsweise die 'Auswertung Anzeigen - Optionen' weiter rechts angezeigt wurde als alle anderen Einstellungen.
		
		\begin{figure}[H]
			\centering
			\includegraphics[width=0.5\textwidth
			]{Images/QuizMaker_Quiz-Settings.PNG}
			\caption{Quiz-Einstellungen Quiz Maker}
		\end{figure}
		
		Aufgeräumter wirkten die Einstellungen beispielsweise bei Quiz Maker \cite{quiz-maker}. Zwar gab es ebenfalls eine Vielzahl von Möglichkeiten, diese wurden aber übersichtlich dargestellt, indem sie Themen zugeordnet und auf Tabs verteilt wurden. Zudem gab es einen eigenen Tab für erweiterte Optionen. \\
		Bei Mobile Quiz waren beide Konzepte möglich, wodurch sich ein Quiz-Ersteller in den Einstellungen schneller zurechtfinden sollte.
		
		
		
		\item Quiz kann vor Veröffentlichung durchgespielt werden \\
		Wie sieht das erstellte Quiz für den Teilnehmer aus? Gibt es noch Rechtschreibfehler oder werden Inhalte nicht optimal dargestellt? Ein Quiz-Ersteller wird sich all diese Fragen womöglich stellen und die einfache Lösung dazu ist, dass man das Quiz vor Veröffentlichung selbst durchspielt. Die eigene Teilnahme soll jedoch nicht zählen, da sie die Auswertungsstatistik verfälschen kann. \\
		Mobile Quiz bot die Möglichkeit an, ein Quiz noch nicht zu veröffentlichen und es auf diese Weise auszuprobieren. Dies funktionierte aber nur, wenn man als Administrator eingeloggt war. Zudem zählte die eigene Teilnahme in die Gesamtauswertung mit hinein. Für die saubere Umsetzung musste also die Funktion für alle Ersteller-Rollen zugänglich sein und die eigene Teilnahme nicht mitzählen.
		
		
		
		\item Template für Frage-Import \\
		Ist man im Zug unterwegs und möchte trotzdem an einem neuen Quiz arbeiten, so fehlt meist der Internetzugang. Dies kompensierte Mobile Quiz dadurch, dass Fragen aus Excel-Dateien eingelesen und erstellt werden konnten. Um dies zu nutzen, benötigte es jedoch eine spezielle Formatierung, was neue Benutzer abschrecken konnte. \\
		Die Lösung dazu war es, ein Excel-Template, also eine Vorlage, für neue Fragen bereitzustellen. Darin kann der Benutzer schnell und einfach Fragen erfassen und muss sich nicht um das Format kümmern. Solche Templates bot beispielsweise
		Socrative \cite{socrative.com} an. Es ist im Anhang unter 'socrativeQuizTemplate.xlsx' zu finden.
		
		Im Rahmen dieser Arbeit wurde ebenfalls ein solches Template erstellt, welches nun Quiz-Erstellern zum Download angeboten wird.
		
	\end{itemize}	
	
		
	\chapter{Konzepte}
	% !TEX root = Projektdokumentation.tex

\newacronym{ICTh}{ICTh}{Informatios- und Codierungstheorie}


%Einleitung mit Erkenntniss / Zusammenfassung aus allen Unterkapiteln
Beim Erstellen der neuen Mockups wurde festgestellt, dass nebst dem Aussehen auch grundsätz-liche Überlegungen zu gewissen Themen vertieft erarbeitet werden sollen. Diese Konzeptüberle-gungen umfassen die Bereiche des Gruppenmanagements, der neuen Fragetypen sowie der Statistiken und den dazugehörigen Auswertungen.
In diesem Kapitel wurde Theorie erarbeitet und dann festgelegt, wie diese Konzepte umgesetzt werden können. Beim Gruppenmanagements wurden die Rollen sowie das Arbeiten mit Gruppen analysiert und erweitert. Im Bereich der neuen Fragetypen wurde zuerst theoretisch neue Frage-Formen erarbeitet und dann die Umsetzung nach Schwierigkeit und Nützlichkeit beurteilt. Schliesslich wurde bei den Statistiken mögliche neue Auswertungsformen untersucht und aufgezeigt, welche neuen Berechnungen dazu vorgenommen werden sollen.

\section{Gruppenmanagement}
In diesem Teil wurden die Rollen sowie das bestehende Gruppenmanagement angeschaut. Zu beiden Themen wurde eine Bestandsaufnahme gemacht, welche dann um sinnvolle Konzepte und Anforderungen erweitert wurde.

\bigskip

Im Bereich Rollen wurden zusätzliche Rollen definiert. Dabei handelt es sich um den Demo-User sowie um den Assistenten.

Die Rolle des Demo-Users ist dazu gedacht, dass jemand anonym die Funktionen von MobileQuiz ausprobieren kann. Nur speziell für den Demo-User gekennzeichnete Quizzes sind dabei ersichtlich und durchführbar. Eine Quiz-Durchführungen des Demo-Users wird in den Statistiken nicht erfasst.

Mit der Rolle des Assistenten wird es für einen Ersteller möglich eine oder mehrere Personen zu bestimmen, welche ebenfalls die Quizzes des Ersteller auswerten, bearbeiten und sogar in seinem Namen erstellen können. Damit wird der Situation Assistent und Dozent aus der realen Welt Rechnung getragen.

\bigskip

Die Idee der bereits bestehenden Gruppen wurde so erweitert, dass ein Teilnehmer neu mehr als einer Gruppe zugewiesen werden kann.

Zudem kann ein Quiz neu mehrere Durchführen haben. Die Festlegung des Durchführungszeitraum sowie des Quiz-Typs wird aus dem Quiz in die Durchführung verschoben. Damit ist es möglich, das Quiz einmal zu erstellen und es dann aufgrund der gewählten Durchführungszeiträume und Quiz-Typen als unterschiedliche Durchführungen laufen zu lassen. Die Durchführung wird so konzipiert, dass ihr entweder eine Gruppe oder ein einzelner Teilnehmer hinzugefügt werden kann. 

\bigskip

Die ausführliche Ausarbeitung aller Ergebnisse ist im Anhang im Dokument \glqq Konzept Gruppenadministration\grqq ab Seite \hyperlink{page.\getpagerefnumber{pdf:gruppenadmin}}{\getpagerefnumber{pdf:gruppenadmin}} ersichtlich. Zudem finden Sie sämtliche Datenbankanpassungen, welche für diese Änderungen notwendig sind, im Dokument \glqq Datenbankmodell Konzept Gruppenadministration\grqq auf Seite \hyperlink{page.\getpagerefnumber{pdf:dbgruppenadmin}}{\getpagerefnumber{pdf:dbgruppenadmin}} sowie im Kapitel Änderungen an der Datenbank \ref{subsec:DBAenderungen}.


\section{Neue Fragetypen}
Für dieses Konzept wurde zuerst die Theorie zu unterschiedlichen Fragetypen erarbeitet. Danach wurde beurteilt, welche neuen Fragetypen wie schwierig umzusetzen sind und wie nützlich sie für MobileQuiz sind. Zudem wurde ein neues Excel-Template entwickelt, dessen detaillierte Beschreibung unter \glqq Frage-Template\grqq  \ref{subsec:FrageTemplate} ersichtlich ist.

\bigskip

\noindent Es wurden folgende Fragetypen als mögliche neue Fragetypen angeschaut:
\begin{itemize}
	\item Single-Choice und Multiple-Choice - Fragen mit Bildern als Antwort
	\item Single-Choice und Multiple-Choice - Fragen mit Bild in der Frage
	\item Freitext
	\item Lückentext
	\item Lückentext mit DropDown Auswahl
	\item Drag \& Drop
	\item Antworten Sortieren
	\item Code-Evaluation
\end{itemize}

\bigskip

Davon wurden die folgenden weiterverfolgt:
\begin{itemize}
	\item Single-Choice und Multiple-Choice - Fragen mit Bild in der Frage:
	Dieser Fragetyp wird beispielsweise bei \acrshort{CN1} verwendet, um Fragen zu einem Netzwerklayout zu stellen.
	\item Freitext:
	Obwohl dieser Fragetyp nicht implementiert wurde, so diente er doch als Inspiration für ein Feedback-Feld unter jeder Frage. Ist eine Frage für einen Studenten nicht verständlich, so kann er seine Frage über dieses Feld direkt an den Quiz-Ersteller senden.
	\item Antworten Sortieren:
	Dieser Fragetyp wäre beispielsweise bei \acrshort{CN1} nützlich, um Netzwerktechnologien nach Geschwindigkeit zu ordnen.
	\item Drag \& Drop:
	Dieser Fragetyp wäre für \acrshort{CN1} und \acrshort{ICTh} attraktiv, beispielsweise für die Bezeichnungen von Übertragungsverfahren.
\end{itemize}

\noindent Für die weiteren Fragetypen besteht derzeit kein Bedarf, weshalb auf die Umsetzung verzichtet wird.

\bigskip

Sämtliche Überlegungen zu diesem Kapitel befinden sich im Dokument \\ 'Strukturierte\_Fragetypen\_AnforderungenV2.docx', die dazu notwendigen Datenbankänderungen sind im Dokument 'newQuestions\_MobileQuizDB.mdj' festgehalten.
Das neu erarbeitete Excel-Template ist unter 'Umsetzungsvorschlag\_Frage\_TemplateV2.xlsx' ersichtlich.

\section{Statistiken und Auswertungen}
In den Statistiken und Auswertungen wurden neue Auswertungstypen angeschaut und beurteilt, welche davon umgesetzt werden sollen. Zudem wurde ausgearbeitet, wie die neuen Auswertungen zugänglich sein sollen.

\bigskip

\noindent Dabei handelt es sich um folgende neue Statistiken:
\begin{itemize}
	\item Mittelwert
	\begin{itemize}
		\item für Punkte
		\item für Zeit
	\end{itemize}
	\item Standardabweichung
	\begin{itemize}
		\item für Punkte
		\item für Zeit
	\end{itemize}
	\item Aufgabenschwierigkeit
	\item Risikobereitschaft
	\item Discrimination Index
	\item Random Guess Score
\end{itemize}

\noindent Ausser dem Random Guess Score werden alle Statistiken weiterverfolgt bzw. sollen umgesetzt werden. Der Aufwand für die tatsächliche Berechnung und Umsetzung des Random Guess Score würde den schlussendlichen Nutzen weit übersteigen.

\bigskip

Die genaue Erläuterung sowie die Berechnung finden Sie im Dokument \\ 'Strukturierte\_Statistiken\_AnforderungV2.docx'. Die dazu nötigen Datenbankänderungen sind im Dokument 'newStatistiken\_MobileQuizDB.pdf' dokumentiert.	
	
	
	\chapter{Software Engineering}
	%Software Engineering: Systembeschreibung, Software-Architektur, Domainmodell, Sequenzdiagramme, etc. (Anhand dieses Teils sollte man genau verstehen, was wie realisiert wurde. Informatiker sollten alle für die Optimierung oder Weiterentwicklung des System nötigen Informationen haben.)
	%Überarbeitung gewisser Konzepte, Bspw. Teilnahme
	% !TEX root = Projektdokumentation.tex


\section{Datenbankänderung}

\subsection{Bestehende Datenbank}

Für die Behebung der bestehenden Fehlern sowie für die Erstellung neuer Funktionalitäten war es wichtig, einen exakten Überblick über die Datenbank zu besitzen. Da zu Beginn nur eine veraltete Übersicht über die Tabellen vorhanden waren, musste diese zuerst aktualisiert werden. Die erstellte Übersicht ist im Anhang unter \glqq Datenbank aktueller Zustand \grqq auf Seite \hyperlink{page.\getpagerefnumber{pdf:dbVorVeraenderung}}{\getpagerefnumber{pdf:dbVorVeraenderung}} zu finden.



\subsection{Änderung an der Datenbank}
\label{subsec:DBAenderungen}
Verschiedene neue Funktionalitäten verlangten eine Änderung der Datenbank, da neue Arten von Daten abgespeichert werden mussten.


\begin{figure}
	\centering
	\begin{subfigure}{.5\textwidth}
		\centering
		\includegraphics[width=0.6\textwidth]{Images/DB_User_Table.PNG}
		\caption{Alte User-Tabelle}
	\end{subfigure}%
	\begin{subfigure}{.5\textwidth}
		\centering
		\includegraphics[width=0.6\textwidth]{Images/DB_Group_Table.PNG}
		\caption{Neue user\_group - Beziehung}
	\end{subfigure}
\end{figure}



\begin{itemize}
	\item Gruppenmanagement\\
	Ein Benutzer konnte bisher nur einer Gruppe zugeordnet sein. Wie auf dem Bild a) ersichtlich, war deren ID direkt in der Tabelle \glqq User\grqq gespeichert. Neu sollte eine Gruppe mehrere Benutzer beinhalten und ein Benutzer in mehreren Gruppen sein können. Es wurde deshalb, wie auf Bild b) ersichtlich, eine Zwischentabelle für die Auflösung der N:M - Beziehung gemacht.\\
	Mehrere Gruppen werden unter anderem dafür benötigt, dass ein Benutzer neu seine\\ Themenbereich-Interessen angeben kann und dafür je einer Gruppe zugeordnet ist. Dazu kommen die Praktikums- und Vorlesungsgruppen.
	\item Quiz-Durchführungen\\
	Da es im Modul \gls{CN1} mehrere Praktikumsgruppen gibt, welche unterschiedliche Zeiträume für die Lösung von Quizzes zur Verfügung haben, benötigte es pro Quiz mehrere Durchführungen.
	
	Neu können mehrere Durchführungen erstellt und zu diesen Gruppen zugeordnet werden. So kann ein Enddatum pro Gruppe festgelegt werden.
	Dafür mussten in der Datenbank einige Änderungen vorgenommen werden. 
	
	Die Wichtigste dieser Änderungen war das Erstellen der \glqq execution\grqq - Tabelle. Die bisherigen Einstellungen für eine Durchführung eines Quizzes wurden aus der Tabelle \glqq questionnaire\grqq in die Tabelle \glqq execution\grqq übertragen.
	
	Es wurden mehrere Verknüpfungstabellen nötig, welche dafür sorgen, dass Quizzes, Teilnehmer und Gruppen einer Durchführung zugeordnet werden können.
	
	Schlussendlich wurde die Tabelle \glqq priority\_settings\grqq erstellt. In dieser werden vom Standard abweichende Einstellungen eines Erstellers abgespeichert. Bei der Erstellung einer neuen Durchführung wird überprüft, ob solche Werte vorhanden sind. Wenn ja, werden diese für die neue Durchführung eingetragen, wenn nein, werden die vordefinierten Standard-Einstellungen von Mobile Quiz verwendet. Das Ziel dahinter ist, dass der Ersteller so wenig wie möglich von Hand einstellen muss, sobald er seinen Durchführungstyp beziehungsweise seine Durchführungspriorität (Lernhilfe, Testatbedingung, Prüfung) gewählt hat.
	
	In der untenstehenden Grafik sind alle Datenbankänderungen festgehalten. Dabei sind die neu erstellten Tabellen orange ausgefüllt und die veränderten Tabellen orange umrahmt.
	
	\begin{figure}[H]
		\centering
		\includegraphics[width=1\textwidth
		]{Images/DB_Execution_Table.PNG}
		\caption{Datenbankänderungen für die Durchführung}
	\end{figure}
	
\end{itemize}

Diese umfangreichen Datenbank-Änderungen zogen viele Anpassungen von Datenbankabfragen nach sich, da beim bisherigen \glqq questionnaire\grqq viele Verbindungen zusammenliefen. So musste von der Quiz-Erstellung, über die Quiz-Durchführung bis zur PDF-Generierung der Auswertung vieles umgeschrieben werden.
	
	
	
	
	\chapter{Realisierung}
	%Realisierung, System- oder Softwarekonzepte
	%Umsetzung von neuen Konzepten und Desgin
	% !TEX root = Projektdokumentation.tex

\newglossaryentry{XML External Entitiy} {name={XML External Entitiy-Attacke},description={}} % TODO Erklären

\section{Verbesserung der bestehenden Lösung}

\subsection{Frage-Template}
Die Möglichkeit, Fragen offline zu erstellen und anschliessend hochzuladen, bestand bereits. Unterstützt wurde die Erfassung von Singechoice und Multiplechoice-Fragen. Dazu konnten Fragen in eine CSV-Datei geschrieben werden. Alle korrekten Antworten wurden mit einem Asterisk (Stern-Zeichen) versehen. Eine Multiplechoice-Frage lag vor, wenn mehr als eine Antwort mit einem Asterisk versehen war.

In dieser Arbeit wurde das CSV-Template durch ein Excel-Template abgelöst. Die Gründe dazu sind die folgenden:
\begin{itemize}
	\item Die Regeln für das Erstellen waren einem kleinen Personenkreis bekannt. Sie wurden auf der Webseite nicht beschrieben. Damit die Funktion aber Verbreitung findet, muss die Vorgehensweise öffentlich zugänglich sein.
	
	Es wurde entschieden, ein Excel-Template zum Download auf der Webseite anzubieten. Darin sind alle relevanten Informationen für die Erstellung vorhanden.
	
	\item Werden die Fragen in Excel erstellt und anschliessend daraus eine CSV-Datei generiert, so hat man die Wahl zwischen 3 verschiedenen CSV-Varianten.
	
	Der Excel-Import hingegen unterstützt alle Excel-Dateien mit der Endung .xlsx. Dieses Format ist ab Excel 2007 das Standardformat und daher weit verbreitet.
	\cite{microsoft2016}
	
	\item Das CSV-Template war auf Singlechoice und Multiplechoice - Fragen beschränkt.
	
	Da neue Fragetypen unterstützt werden sollten, wurde eine neue Struktur mittels Excel erarbeitet. Um den Ersteller zu Unterstützen wurde mit Farben und weiteren Funktionen gearbeitet, welche durch CSV nicht unterstützt werden.
\end{itemize}

Ein Beispiel des ursprünglichen Formats sowie des neuen Excel-Templates ist im Anhang, im Kapitel 19 Details zur Lösungsfindung, ersichtlich.

Der neue Template-Import wurde mit PHPExcel \cite{phpexcel} umgesetzt. Diese PHP-Library war bereits im Projekt eingebunden und wird dazu verwendet, die Rangliste aller Teilnehmenden eines Quiz zu Exportieren. Die neue Logik befindet sich hauptsächlich in der neu erstellen Datei 'importExcel.php'.

Bei PHPExcel gab es bis zur Version 1.7.9 eine \gls{XML External Entitiy} - \gls{Vulnerability}. Dadurch war es Remote-Angreifern möglich, beliebige Dateien auf dem Server zu lesen oder eine Denial-of-Service - Attacke durchzuführen. \cite{cvedetails_phpexcel} %TODO Remote und Denial-of-Service erklären
Da bei MobileQuiz allerdings die Version 1.8.0 verwendet wird, ist dies nicht mehr möglich. Es wurde mit dem Wissen von InfSi3 versucht eine solche Attacke durchzuführen, was ebenfalls zum Ergebnis führte, dass diese Sicherheitslücke nicht mehr vorhanden ist. %TODO InfSi3 erklären

\section{Neuerungen}

\subsection{Fragen mit Bildern}

\subsubsection{Hochladen und Entfernen eines Bildes}
Bei jedem Fragetyp ist es neu möglich ein Bild zu hinterlegen. Die Funktion des Bild-Uploads auf den Server bestand bereits, wurde aber erweitert.
\\
\\
Beim Hochladen des Bildes wurde zuvor eine Fehlermeldung ausgegeben, wenn ein Bild eine Grösse von über 20 Megabyte überstieg. Diese Limite wurde herabgesetzt, weil ein Bild von dieser Grösse lange benötigt, bis es heruntergeladen werden kann. Unten ist ein Vergleich von Downloadzeiten aufgeführt, bei verschiedenen Dateigrössen. Die Downloadgeschwindigkeit ist die durchschnittliche Downloadgeschwindigkeit der Swisscom-Kunden, welche anhand von Speedtests der cnlab AG \cite{cnlab_speedtest} ermittelt wurde. Es wurden die Swisscom-Kunden gewählt, weil die Anzahl der User dort am grössten war. \\

\begin{tabular}{|c|c|c|}
	\hline 
	Dateigrösse & Downloadgeschwindigkeit & Benötigte Zeit \\ 
	\hline 
	20 MB & 33,1 Mbit/s & 4,8 s \\ 
	\hline 
	5 MB & 33,1 Mbit/s & 1,2 s \\ 
	\hline 
	800 KB & 33,1 Mbit/s & 0,2 s \\ 
	\hline 
\end{tabular}\\

Falls ein Bild nun 800 Kilobyte übersteigt, so wird es so lange komprimiert, bis seine Dateigrösse darunterliegt. Dies wird durch die PHP-Funktion 'imagecopyresampled' erreicht. Als Code-Vorlage diente das Beispiel von www.williseiler.ch \cite{willis_php}. \\

Weiter gibt es Grössenbeschränkungen von Apache selbst \cite{stackoverflow_largeFilePHP}. Im File 'php.ini' können folgende drei Werte definiert werden:
\begin{itemize}
	\item upload\_max\_filesize: Legt die maximale Dateigrösse für eine Upload-Datei fest. Der Standardwert liegt bei 2 Megabyte.
	\item post\_max\_size: Legt die maximale Grösse aller POST-Daten zusammen fest. Der Standardwert liegt bei 8 Megabyte.
	\item max\_file\_uploads: Legt die maximale Anzahl aller Dateien fest, welche auf einmal hochgeladen werden können. Der Standardwert liegt bei 20 Dateien.
\end{itemize}

%TODO: Ergebnis der Anpassung festhalten

Beim Upload wird das mit folgendem Dateinahmen im Ordner 'uploadedImages' abgelegt: 'question\_Datum(Tag\_Monat\_Jahr\_Stunde\_Minute\_Sekunde\_\_SessionId.Dateityp'.
\\
\\
Das Excel-Template für die Frage-Erstellung wurde ebenfalls erweitert, dass auch Fragen mit Bildern erstellt werden können. Dazu wurde eine neue Spalte eingefügt, bei welcher man den Bildnamen inklusive Endung einträgt. Die Bilder müssen sich dabei im gleichen Ordner wie das Template befinden. Anschliessend wird der gesamte Ordner hochgeladen.

Es war dabei nicht möglich, nur die Excel-Datei alleine hochzuladen und anschliessend die Bilder selbstständig zu holen. Dies wäre bezüglich der Sicherheit höchst bedenklich.
\\
\\
Wird eine Frage gelöscht, so wird nebst dem Datenbankeintrag auch das Bild vom Server entfernt.


\subsubsection{Änderungen am Server}
\textbf{Wichtig:} 
Für den Server-Upload müssen die folgenden zwei Dinge gegeben sein:
\begin{itemize}
	\item Die PHP-Library 'GD' muss für die Bildverarbeitung installiert sein. Ob dies der Falls ist kann mittels PHPInfo() oder nachfolgendem Code festgestellt werden. \cite{zoopable.com} Sollte 'GD' nicht installiert sein, so ist unten ebenfalls der Installationsbefehl aufgeführt. \cite{askubuntu.com_php_extension}
\end{itemize}
\begin{lstlisting}
<?php
if (extension_loaded('gd') && function_exists('gd_info')) {
echo "PHP GD library is installed on your web server";
}
else {
echo "PHP GD library is NOT installed on your web server";
}
?>

sudo apt-get install php5.6-gd
\end{lstlisting}

\begin{itemize}
	\item Die Childprozesse des Apache-Servers, welche die Anfragen beantworten, benötigen Schreibrechte auf den Upload-Ordner. Ob dies gegeben ist, kann mittels folgendem Befehl überprüft und geändert werden. \cite{askubuntu.com_permissions} 
\end{itemize}
\begin{lstlisting}
ps -ef | grep apache | grep -v grep
\end{lstlisting}

\begin{lstlisting}
root      5001     1  0 07:21 ?    00:00:00 /usr/sbin/apache2 -k start
www-data  5021  5001  0 07:21 ?    00:00:00 /usr/sbin/apache2 -k start
www-data  5022  5001  0 07:21 ?    00:00:00 /usr/sbin/apache2 -k start
www-data  5023  5001  0 07:21 ?    00:00:00 /usr/sbin/apache2 -k start
\end{lstlisting}

Ist die Ausgabe wie oben ersichtlich, so werden die Anfragen von Prozessen der Benutzergruppe 'www-data' beantwortet. Die Schreibrechte an diese Gruppe können wie folgt vergeben werden:
\begin{lstlisting}
chgrp www-data /path/to/mydir
chmod g+w /path/to/mydir
\end{lstlisting}




\subsubsection{Anzeige eines Bildes}
Das Bild sollte beim Anzeigen der Frage gross genug dargestellt werden. Auf dem Computer war dies kein Problem, da ein Browserfenster genug Platz dazu bot, bei einem Smartphone hingegen war dieser beschränkt. Deshalb wurde die Anzeige so umgesetzt, dass das Bild beim Klick darauf auf dem ganzen Bildschirm dargestellt wird. Dies wurde mit Photoswipe \cite{photoswipe} umgesetzt, welches von GitHub \cite{github_photoswipe} bezogen wurde. Photoswipe ist eine JavaScript-Library, welche es ermöglicht, Bildgalerien auf Websites darzustellen. Sie unterstützt unter anderem Touch-Gesten und Zoom. So ist es kein Problem, ein Bild mit vielen Details darzustellen.

Nachfolgend ist zu sehen, wie eine Frage mit Bild in verschieden Situationen angezeigt wird.

\begin{figure}[H]
	\centering
	\includegraphics[width=0.75\textwidth]{Images/Frage-Bild_Anzeige_PC.PNG}
	\caption{Anzeige des Frage-Bildes am Computer}
	Quelle: Mobilequiz.ch / Fragebild: https://pixabay.com/de/mario-luigi-figuren-lustig-bunt-1558012/
\end{figure}

\begin{figure}[H]
	\centering
	\includegraphics[width=0.3\textwidth]{Images/Frage-Bild_Anzeige_Mobile.PNG}
	\caption{Anzeige des Frage-Bildes auf dem Smartphone}
	Quelle: Mobilequiz.ch / Fragebild: https://pixabay.com/de/mario-luigi-figuren-lustig-bunt-1558012/
\end{figure}

\begin{figure}[H]
	\centering
	\includegraphics[width=0.3\textwidth]{Images/Frage-Bild_Anzeige_Mobile_Full.PNG}
	\caption{Fullscreen-Anzeige des Frage-Bildes auf dem Smartphone}
	Quelle: Mobilequiz.ch / Fragebild: https://pixabay.com/de/mario-luigi-figuren-lustig-bunt-1558012/
\end{figure}


Weiter wurde die Generierung des PDF-Aufgabenblattes und Lösungsblattes ergänzt, damit auch dort die Bilder angezeigt werden. Hat man unterwegs kein Internet, so kann man sich die Fragen damit auch vorgängig herunterladen und im Zug anschauen.




	
	\chapter{Inhalte für Folgearbeiten}
	% !TEX root = Projektdokumentation.tex

\section{Arbeiten für die cnlab AG}






\section{Inhalte für weitere Studentenarbeiten}
\label{sec:InhalteFuerStudentenarbeiten}

\begin{itemize}
	\item Umgestaltung der Willkommensseite \\
	Wie im Kapitel \ref{subsec:Webuntersuchungen} beschrieben, besteht Verbesserungspotential beim ersten Eindruck von Mobile Quiz. Ein entsprechender Blog-Eintrag zu diesem Thema ist ebenfalls dort aufgeführt.
	
	\item Durchspielen vor Veröffentlichung \\
	Um ein Quiz auf Fehler zu Überprüfen soll es vor Veröffentlichung durchgespielt werden können. Dies ist bereits möglich, indem das Quiz auf privat gesetzt wird. Dies funktioniert aber nur im Administratoren-Modus korrekt.
	Wird das Quiz anschliessend veröffentlicht, soll der Ersteller gefragt werden, ob er die Quiz-Statistik zurücksetzen will. Ohne die Mitzählung seiner eigenen Teilnahme erhält er schlussendlich eine aussagekräftigere Auswertungsstatistik.
	
	\item Anonyme Teilnahme \\
	Ein Teilnehmer eines Quizzes soll sich nicht anmelden müssen. Eine Anonyme Teilnahme am Quiz soll möglich sein. Speziell daran ist die Handhabung von mehreren Sessions nicht erstellter Benutzer.
	\todo{Frage: Um Statistiken nicht zu verfälschen soll es die Rolle «Test-User» geben, dessen Teilnahme nicht mitgezählt wird. Wurde wieder verworfen?}
	
	\item Sicheres Login und Passwort-Recovery
	Das Login sowie das Zurücksetzen des Passwortes sollen mittels OWASP Cheat-Sheets umgesetzt werden. Diese sind unter folgenden Links zu finden:
	\begin{itemize}
		\item Login \\
		\url{https://www.owasp.org/index.php/Authentication_Cheat_Sheet}
		\item Password-Recovery \\ \url{https://www.owasp.org/index.php/Forgot_Password_Cheat_Sheet}

		
	\end{itemize}
	
	\item Speicherung der Filter-Einstellungen \\
	Wählt der Benutzer eine Einstellung, ein Filter oder ähnliches, welches vom Standard-Fall abweicht, so soll dies in seinen Benutzerdaten gespeichert werden. In dieser Arbeit wurde bereits umgesetzt, dass der Teilnehmer ein oder mehrere Interessengebiete angeben konnte. Wechselte er auf die Ansicht aller Quizzes, so wurde zuerst nach diesen Interessengebieten gefiltert.
	
	
	\todo{Mögliche Arbeiten SA durchgehen}
	
	\todo{Konzepte durchgehen}
	
	\item Statistiken
	\begin{itemize}
		\item Statistiken pro Gruppe
		\item Der Quiz-Ersteller soll bei einem Testat sehen, wer von der Gruppe das Testat bestanden hat und wer nicht. (z.B. Rot/Grün eingefärbt)
		\item Die Ergebnisse, welche eine Gruppe über mehrere Quizzes hinweg erreicht hat, sollen als Verlauf dargestellt werden. So kann P. Heinzmann prüfen, wie die CN1-Teilnehmer über das Semester hinweg die Lernhilfen der Vorlesung gelöst haben.
		
	\end{itemize}
	
	\todo{Mockups durchgehen}
	\item Mockups umsetzen \\
	\begin{itemize}
		\item Quiz-Auswertungs-Ansicht \\
		In der Ansicht von «Auswertung» die Anzeige der gewählten und richtigen Antwort links vor den Antworttexten stehen. Dies hat den Vorteil, dass die Auswahl und der Antworttext näher beieinander liegen.
		Ebenfalls soll darin die Funktion vorhanden sein, um die Auswertung als PDF zu speichern.
		\item Erfassung von Gruppen \\
		In der Ansicht \glqq Quiz Erstellen – Administration\grqq soll die Möglichkeit bestehen, Gruppen zu erfassen. Wird ein Quiz anschliessend veröffentlicht, so wird diese Gruppe benachrichtigt.
		\item Anzeige von wichtigen Quizzes \\
		Ziel ist es, dass einer Gruppe ein Quiz, beispielsweise als Testat, zugewiesen werden kann. Anschliessend sieht der Teilnehmer auf der Quiz-Übersichts-Seite sofort, welche Quizzes seine sofortige Bearbeitung benötigen.
		Im Optimalfall sieht der Teilnehmer durch diese Option sowie durch das automatische Setzen des Filters nach seinen Interessen bereits alle Quizzes, welche er benötigt.
	\end{itemize}
	
	\item Automatisch Erfassung von Teilnehmern und Gruppen \\
	Die Erfassung aller Teilnehmer für eine Vorlesung soll durch den Administrator erfolgen können. Dieser zu Beginn des Semesters alle Studenten via einer Excel-Datei ein, welche er von \url{www.unterricht.hsr.ch} generieren liess. Mobile Quiz erstellt automatisch die Teilnehmer. Auf einer Gruppen-Management-Seite weist der Administrator die Teilnehmern dann den Gruppen zu.
	
	\item Durchführungsbeschränkungen \\
	Es soll möglich sein zu unterscheiden, ob nur zugewiesene Gruppen ein Quiz durchführen können oder ob das Quiz allen offenstehen soll.
	
	\item Umsetzung von Polls \\
	Als Vorlage für die bestehende Lösung durch Patrick Eichler diente \glqq Straw Poll\grqq . \cite{straw_poll} 
	%Email-Content von M. Stolze hinzufügen
	Dazu sind Grundlegende Abklärungen zu Umfragen zu erarbeiten.
	
	\item Session pro Benutzer \\
	Die Session soll pro Benutzer und nicht pro Gerät erstellt werden. Ist ein Benutzer am Computer angemeldet und meldet sich anschliessend auch auf dem Smartphone an, so soll die Computer-Session beendet werden.
	
	\item Schlagwörter \\
	
	\item Frage-Einreichung von Teilnehmer \\
	Es gibt eine neue Funktion, mit der Teilnehmer eigene Vorschläge für Fragen und Antworten an den Quiz-Ersteller senden können. Der Ersteller des Quizzes kann dann die Vorschläge annehmen, ablehnen oder anpassen.
	
	\item Ursprungsseite beim Melden von Fehlern feststellen \\
	Beim Beheben von Fehlern wäre es nützlich zu wissen, auf welcher Seite sich der Benutzer befand, bevor er auf auf «Fehler melden» klickte.
	
	\item Status-Konzept überarbeiten \\
	Konzept der Status soll dabei überarbeitet werden: Welche gibt es genau und was verursacht ein Status-Übergang?
	Bisherige Status: offen (0\%), in Bearbeitung(1-99\%), erledigt(100\%)
	\todo{Nachschauen, welche Status vorhanden}
	
	
	
\end{itemize}



	
	
	\chapter{Qualitätsmanagement}
	% !TEX root = Projektdokumentation.tex

\newglossaryentry{Usability-Tests}{name={Usability-Tests},description={Probanden aus der Zielgruppe der Anwendung werden Aufgaben gestellt, welche sie mit der bestehenden Anwendung lösen sollen. Dabei wird untersucht, welcher Weg zur Lösung der Aufgabe eingeschlagen wird und wo dabei Probleme auftauchen.}}

\newacronym{CN1}{CN1}{Computernetze 1}

\newglossaryentry{Wireframes}{name={Wireframes},description={Die Visualisierung stellt die Seitenstruktur und Featureumsetzung sehr grob und schematisch dar. Der Wireframe wird in schwarz-weiss-grau angefertigt und gleicht dadurch einer Skizze oder Bleistiftzeichnung. Dieses Art der Visualisierung ist sehr schnell, einfach und günstig zu erstellen. Dazu genügt Papier und Bleistift oder eine entsprechende Wireframe-Software.}}

\newacronym{UI}{UI}{User Interface}

\newglossaryentry{User Interface}{name={User Interface},description={Unter einer Benutzeroberfläche oder Benutzerschnittstelle (UI) versteht man die Art und Weise, wie Befehle und Daten in den Computer eingegeben werden. Die Benutzeroberfläche ist die Schnittstelle zwischen Computer und Mensch. \cite{itWissen_benutzeroberflache}}}

%Qualitätsmanagement: Messungen, Tests, Usability Tests, Code Review usw. (mit vollständiger Beschreibung der Anordnungen und Rahmenbedingungen)

\section{Usability}
Im Internet gibt es zahlreiche Online-Quizzes, auch für das schulische Umfeld. Damit Mobile Quiz häufig und gerne genutzt wird, gibt es einige Faktoren zu beachten. \cite{marketingfire.de} Dazu zählen unter anderem das Design und die Strukturierung der Seite. \\

Wie gut die bestehende Mobile Quiz - Version in diesen Bereichen abschneidet, kann mit einem \gls{Usability-Tests} festgestellt werden. Davon wurden zwei Durchführungen gemacht, wobei die erste zu Beginn der Arbeit dabei half, Schwierigkeiten in der Bedienung offenzulegen. Anschliessend flossen die Ergebnisse draus in die Aufgabenstellung mit ein. Gegen Ende der Arbeit fand dann die zweite Durchführung statt, um zu messen, welche Fortschritte durch die Arbeit gelungen sind.

\subsection{Methoden}
Bei den \gls{Usability-Tests} zu Beginn der Arbeit nahmen drei Studenten der \acrfull{CN1}-Vorlesung, ein Student aus der Raumplanung sowie ein Student aus dem 5. Semester Informatik teil, was der Zielgruppe von Mobile Quiz entspricht. Zudem hatten die Studenten aus \acrshort{CN1} erst wenig Erfahrung damit gesammelt. \\
Bei der Durchführung wurden die Teilnehmern in Situationen hineinversetzt, welche bei der Benutzung von Mobile Quiz oft vorkommen (siehe Usability-Test\_Aufgabenstellung). Die Teilnehmer wurden dabei eins zu eins beobachtet und Schwierigkeiten oder Abweichungen von den Erwartungen (siehe Usability-Test\_Erwartungen) notiert. Die Gesamtauswertung wurde anschliessend in einem separaten Dokument festgehalten (siehe Usability-Test\_Auswertung). Die erwähnten Dokumente befinden sich im Anhang.

\subsection{Erkenntnisse}
% Hier folgen sämtliche Erkenntnisse zum Bereich Usability.
% Auch die Auswertung was neu gemacht wird, kommt hier hin bzw. sicher eine Zusammenfassung und ganzes ist dann im Anhang.
Die Durchführung der \gls{Usability-Tests} zeigte, dass vor allem im Bereich der Benutzerführung Probleme vorhanden sind, denn die vorhandenen Funktionen werden nicht auf den ersten Blick gefunden.
Aus den Erkenntnissen der \gls{Usability-Tests} sowie den eigenen Tests mit dem MobileQuiz wurden die ersten \gls{Wireframes} erstellt, diese befinden sich im Anhang. 

\section{Codestatistik}
Mit Codestatistiken soll erkannt werden, ob sich die Codequalität während der Arbeit verbessert oder nicht. Dafür wird das Webtool Code Climate eingesetzt.

Der nachfolgende Screenshot zeigt den Stand der Codequalität ganz zu beginn des Projektes. Die bereits eingezeichneten Verbesserungen sind entstanden, da die Grundeinstellungen auf unsere Bedürfnisse angepasst und verfeinert wurden.

\begin{figure}[H]
	\centering
	\includegraphics[width=1\textwidth
	]{Images/Stand_Beginn_181016.PNG}
	\caption{Code Climate Stand 18.10.2016 - nach der Verfeinerung der Regeln}
\end{figure}


\section{Systemtests mit Selenium}
%automatisierte Systemtests mit Selenium
Selenium IDE ist ein Firefox AddOn für Web-\acrfull{UI}-Tests. Es ermöglicht das Aufnehmen, die Bearbeitung, das Debuggen und das Abspielen von Tests. 

Mit der Hilfe dieses Tools können einfach \gls{User Interface} Tests durchgeführt werden.

\subsection{Methoden}
%Konkretes Vorgehen beschreiben
Mit dem Firefox Plugin von Selenium können die Abläufe die getestet werden sollen einfach aufgenommen werden. Das heisst der Benutzer spielt den korrekten Ablauf durch. Das einzige was von Hand gemacht werden muss, ist das assert-Statement, also die Prüfung, ob der Test korrekt durchgelaufen ist.

Alle erfassten Tests werden abgespeichert und können danach einfach abgespielt werden.

\subsection{Erkenntnisse}
%Was haben wir mit Hilfe von (siehe Titel) herausgefunden und wie werden wir es verbessern?
%Folgen sobald das Tool konkret eingesetzt wurde

\section{Code Review}
%Code Review mit GitHub Branch
%Einleitung
Mit einem Code Review wird die Codequalität sichergestellt. Dabei schaut sich jeweils derjenige, welcher den Code nicht geschrieben, den Code des anderen an. 


\subsection{Methoden}
%Konkretes Vorgehen beschreiben
Die Code Reviews wurden mit Github Branches gemacht. Das heisst der Entwickler eröffnet für seine neuen Features ein Github Branch und bevor dieser wieder in den Master merged werden kann, schaut sich das andere Teammitglied den Branch an und gibt in frei.

\subsection{Erkenntnisse}
%Was haben wir mit Hilfe von (siehe Titel) herausgefunden und wie werden wir es verbessern?
%Folgen sobald konkret eingesetzt wurde

\section{Unit-Tests}
%Unit-Tests mit PHPUnit
%Einleitung
Die Unit-Tests sind dazu da, einzelne Funktionen zu testen. Wenn eine Basis von Unit-Test bestehen, welche die Funktionalität des Codes abdecken, kann gut Refactoring betrieben werden. 

\subsection{Methoden}
%Konkretes Vorgehen beschreiben
Das Testen wurde mit PHPUnit umgesetzt. Diese Test werden bei jedem commit automatisch vom Continuous Integration Server, in unserem Fall Travis CI, ausgeführt. So wird bei jedem commit geschaut, ob noch alles funktioniert

\subsection{Erkenntnisse}
%Was haben wir mit Hilfe von (siehe Titel) herausgefunden und wie werden wir es verbessern?
%Folgen sobald konkret eingesetzt wurde



	
	
	\chapter{Schlussfolgerung}
	%Die Schlussfolgerungen bilden zusammen mit der Kurzzusammenfassung (Abstract) und der Zusammenfassung (Mgmt Summary, Broschürentext) die wichtigsten Abschnitte eines Berichts und sollen daher am sorgfältigsten ausgearbeitet sein.
	
	%Die Schlussfolgerungen enthalten eine Zusammenfassung und Beurteilung der Resultate (Vergleich mit anderen Lösungen, was wurde erreicht, was nicht, was bleibt noch zu tun, was würde man nun anders tun).
	%Bei Systementwicklungen kann eine Vergleichstabelle sinnvoll sein, welche die wichtigsten Funktionen im Vergleich zu verschiedenen Lösungen zeigt.
	%In den Schlussfolgerungen soll auch ein Ausblick auf das weitere Vorgehen bzw. auf die Bedeutung der erreichten Ergebnisse gegeben werden.
	% !TEX root = Projektdokumentation.tex

\section{Zusammenfassung}
Im Rahmen dieser Studienarbeit wurden die bereits bestehende Mobile Quiz Anwendung um neue Funktionalitäten erweitert und Benutzerfreundlicher gemacht. Wobei ein grosser Teil der Veränderungen Überarbeitung des bestehenden Mobile Quizzes war und nur ein kleiner Anteil die Ausarbeitung von neuen Funktionalitäten.

\bigskip

Die verbesserte Benutzerfreundlichkeit konnte dadurch erreicht werden, in dem die Inputs aus den \gls{Usability-Test}s und den eigenen Untersuchgen in die Erstellung der neuen Seitendesigns mit einflossen. Die Bestätigung, dass die vorgenommen Änderungen zu einer benutzerfreundlicheren Bedienung führte, ergab sich mit den zum Schluss durchgeführten \gls{Usability-Test}s. Die zusätzliche Fehlerbehebungen zu Beginn der Studienarbeit erhöhte die Zuverlässigkeit von Mobile Quiz ebenfalls.

\bigskip

Die neuen Funktionalitäten ergänzen die Online Quiz-Plattform in verschiedenen Bereichen. Mit den Bild-Fragen ergeben sich für den Ersteller eines Quizzes neue Möglichkeiten, Fragen zu stellen. Auch der Teilnehmer profitiert von dieser Neuerung, denn das Lösen eines Quizzes wird für ihn spannender.
Mit der Möglichkeit, während dem Lösen eines Quizzes, den Ersteller des Quizzes zu kontaktieren, sollen Unklarheiten möglichst schnell aus dem Weg geschafft werden.

\bigskip

Alles in allem konnte Mobile Quiz durch die Umsetzung der oben beschriebenen Punkte zu einer verlässlicheren und bedienungsfreundlicheren Quiz-Plattform umgestaltet werden.

\section{Ausblick}
Die neu Entwickelte Version wird ab dem nächsten Semester produktiv eingesetzt. Vor der effektiven Einsetzung müssen allerdings noch einige kleinere Arbeiten erledigt werden, für die es zum Schluss der Studienarbeit leider nicht mehr reichte.

\bigskip

Zum einen muss die Implementation der Durchführung noch fertiggestellt werden, denn ohne die fertige Implementation, können keine gültigen Durchführungen erstellt werden. Zudem gibt es eine Liste mit vielen kleineren Aufgaben zur Weiterentwicklung, welche durch die HSR erledigt werden sollten. Darunter befinden sich Aufgaben wie:
\begin{itemize}
	\item Fehlermeldung bei zu grossen Bildern
	Wenn ein zu grosses Bild hochgeladen wird, wird dieses vom Server nicht angenommen. Zur Zeit wird dem Ersteller eine allgemeine Fehlermeldung angezeigt. Diese Fehlermeldung müsste so angepasst werden, dass der Ersteller das Problem auf den ersten Blick erkennt.
	\item Umsetzung der Verbesserungsvorschläge aus den zweiten \gls{Usability-Test}s
\end{itemize}

\noindent Die vollständige Liste befindet sich im Kapitel \ref{subsec:NichtFertiggestellteArbeiten}.

\bigskip

Es wurden auch Konzepte ausgearbeitet, für welche es in der Implementierungsphase keine Zeit mehr hatte. Deshalb ist es gut möglich, dass es eine weitere Studienarbeit für Mobile Quiz geben kann. Darin könnten dann zum Beispiel die in der in der Analysephase ausgearbeiteten Statistiken umgesetzt werden. Es wurden zudem bereits einige Mockups erstellt, für welche die Zeit nicht mehr reichte um diese umzusetzen.

\bigskip

\noindent Eine vollständige Liste aller Ideen für eine folgende Studienarbeit befindet sich im Kapitel \ref{sec:InhalteFuerStudentenarbeiten}.
	
	
	\chapter{Literaturverzeichnis}
	%Im Literaturverzeichnis sind alle verwendeten Quellen (Bücher, Publikationen, Application Notes, Links (url), sowie Hinweise auf Gespräche mit bestimmten Personen) aufgeführt. Typischerweise werden die Quellenangaben nummeriert (z.B. [1], [2]) und in der Reihenfolge geordnet, wie sie im Bericht vorkommen. Man kann die Quellen auch mit den (abgekürzten) Namen der Autoren und dem Erscheinungsjahr bezeichnen (z.B. [Schueli90], [Shannon49]), wobei die Einträge dann alphabetisch geordnet werden. Jede Referenz ist so anzugeben, dass sie einfach auffindbar ist (evtl. inklusive Seitenzahl, Bibliothek Bestellnummer). Eine Referenz, welche die allgemeine Grundlage für ein ganzes Kapitel bildet, wird im Titel bzw. in einer Fussnote aufgeführt. Für Referenzen aus dem Internet soll ein kommentierter URL angegeben werden.
	\patchcmd{\thebibliography}{\chapter*}{\section*}{}{}
	\renewcommand{\bibname}{}
	\bibliographystyle{IEEEtran}
	\bibliography{zotero}
	
	
	\chapter{Abkürzungsverzeichnis (Glossar)}
	%Das Abhkürzungsverzeichnis (Glossar) enthält alle im Bericht vorkommenden Abkürzungen in alphabetischer Reihenfolge. Häufig wird hier auch eine Kurzbeschreibung zu den Begriffen angegeben. In diesem Fall nennt man den Abschnitt eher "Glossar".
	\renewcommand{\glossarysection}[2][]{}
	\printglossaries
	
	
	\chapter{Abbildungsverzeichnis}
	\renewcommand{\listfigurename}{}
	\listoffigures
	
	
	\part{Anhang}
	%Der Anhang enthält Abschnitte, welche den Rahmen oder die Kontinuität der Hauptkapitel stören, da sie nicht von zentraler Bedeutung für die Endlösung sind. Auch administrative Teile der Arbeit, welche von der Schule gefordert, für das Verständnis des Endergebnisses der Arbeit weniger wichtig sind, sollen im Anhang aufgeführt sein. Die im folgenden hervorgehobenen Punkte müssen in jeder Arbeit enthalten sein, die übrigen Punkte können je nach Situation vorkommen oder nicht.
	
	
	\chapter{Persönliche Berichte zur Arbeit}
	%Persönliche Berichte zur Arbeit (wie von HSR verlangt)
	\section{Persönlicher Bericht von Andrea Hauser}
	% !TEX root = Projektdokumentation.tex

Da wir das Thema unserer Studienarbeit auf eigenen Wunsch mit einbringen konnten, habe ich mich schon vor dem Semesterbeginn sehr darauf gefreut, diese Arbeit anzupacken. Dementsprechend hoch war auch die Motivation, die Studienarbeit anzugehen. Mit voller Überzeugung kann ich sagen, dass die Freude und die Motivation durchgehend bestehen geblieben sind. 

\bigskip

Die vierzehn Wochen, in welchen diese Studienarbeit bearbeitet wurde, sind wie im Flug vergangen. Dies lag sicher auch daran, dass wir mit insgesamt 36 ECTS einen gut gefüllten Stundenplan und sehr viele Abgaben hatten.
Auch wenn es teilweise stressige Zeiten gab und ich dann zum Schluss mit meinem Implementierungsteil nicht mehr ganz fertig geworden bin, kann ich sagen, dass ich mit dem Schlussresultat unserer Arbeit zufrieden bin.

\bigskip

Ich habe das Gefühl, dass ich anhand dieser Arbeit sehr viel mehr lernen konnte, als wenn ich das ganze nur in der Theorie angeschaut hätte. Ich hatte vorher noch nie in PHP programmiert und auch den ganze Umgang mit Servern kannte ich vorher noch nicht. Auch wenn ich bei der Zeiteinschätzung der Arbeitspakete manchmal stark daneben lag, hilft mir dies, in meinen in Zukunft anfallende Arbeiten realistischere Zeitschätzungen zu erhalten. Ich bin froh, dass ich diese Erfahrungen bereits hier im Studium sammeln konnte.

\bigskip
Zusammenfassend kann ich sagen, dass ich von dieser Studienarbeit nur profitieren konnte. Ich habe neben dem Engineering Projekt nun ein weiteres Projekt erfolgreich durchführen können und hoffe, dass ich für die Zeiteinschätzung weiter hinzugelernt habe.

	\newpage
	\section{Persönlicher Bericht von David Windler}
	% !TEX root = Projektdokumentation.tex

Die Studienarbeit ist mein zweites grosses Informatik-Projekt im Studium und auch in meiner bisherigen Laufbahn. Da ich zuvor eine KV-Lehre absolvierte und als Quereinsteiger ins Informatik-Studium startete fehlte mir vor allem die Praxiserfahrung. Trotz noch so vielem Lernaufwand konnte diese nie weggemacht werden. Aus diesem Grund bin ich sehr froh um die drei Praktischen Projekte während des Studiums - das Engineering-Projekt, die Studienarbeit und die Bachelorarbeit. Durch diese lerne ich den theoretisch erlernten Stoff des Studiums praktisch umzusetzen und Wissenslücken selbstständig zu schliessen.

\bigskip

Zu Beginn dieser Arbeit hatte ich mit vielen der eingesetzten Technologien von Mobile Quiz noch nie gearbeitet und sehr grossen Respekt davor. Kann ich wirklich selbst einen Apache-Server aufsetzen? Wie bekomme ich die Webseite auf den Server? Alles Fragen, welche ein Informatik-Lernender leicht hätte beantworten können, ich jedoch nicht wusste.
Diese und viele weitere Fragen beschäftigten mich, bis Andrea und ich, mit grosser Unterstützung von Patrick Eichler, Apache auf einer Server-Instanz der HSR zum laufen brachten. Von da an wusste ich, dass auch die restlichen Technologien mit genügend Ehrgeiz zu erlernen sind. Dies bewahrheitete sich auch, denn das Erlernen PHP und Latex bereiteten keine Mühe und auch die SQL-Kenntnisse aus dem ersten Semester konnten schnell wieder in Erinnerung gerufen werden.

\bigskip

Während des Projektes war ich vom vielen Recherche- und Schreibaufwand überrascht und freute mich deshalb umso mehr auf das Implementieren der erarbeiteten Konzepte. Diese Phase fiel jedoch mit vielen Abgaben von anderen Modulen zusammen, wodurch sich die Stunden am Computer anhäuften. Trotzdem machte es viel Spass zu sehen, wie schnell man neue Programmiersprachen lernt, wenn man sie nur oft genug verwendet.

\bigskip

Kurz zusammengefasst: Eine intensive, aber sehr spannende und lehrreiche Zeit.
	
	
	\chapter{Details zur Lösungsfindung}
	%Details zur Lösungsfindung (z.B. Analyseschritte, Designschritte, umfangreiche Herleitungen, Tabellen, Messergebnisse, Testauswertungen)
	
	\includepdf[pages={1-5}, pagecommand={}]{PDFs/Ergebnisse_eigene_Tests.pdf}
	\includepdf[pages={1-6}, pagecommand={}]{PDFs/Gruppenadministration.pdf}
	\includepdf[landscape=true]{PDFs/Gruppenadministration_DB.pdf}
	\includepdf[pages={1-13}, pagecommand={}]{PDFs/Neue_Fragetypen.pdf}
	\includepdf[pages={1-12}, pagecommand={}]{PDFs/Statistiken_Auswertungen.pdf}
	\includepdf[pages={1}, landscape=true, pagecommand={\label{pdf:moeglicheArbeiten}}, pagecommand=\thispagestyle{mylandscape}]{PDFs/MoeglicheArbeitenSA.pdf}
	\includepdf[pages={2-3}, landscape=true, pagecommand=\thispagestyle{mylandscape}]{PDFs/MoeglicheArbeitenSA.pdf}
	\includepdf[landscape=true]{PDFs/CSV_Template.pdf}
	\includepdf[landscape=true]{PDFs/socrativeQuizTemplate.pdf}
	\includepdf[landscape=true]{PDFs/Excel-Frage-Template.pdf}
	\includepdf[pages={1-3}, pagecommand={}]{PDFs/Recherchetipps_book-a-librarian.pdf}
	
	
	\chapter{Mockups}
	\label{chap:mockups}
	Alle nachfolgenden Mockups wurden mit dem Online-Mockup-Tool myBalsamiq erstellt. Die Lizenz wurde von der HSR zur Verfügung gestellt.
	
	Durch die nachträgliche Besprechung mit dem Betreuer wurden die nachfolgenden  Änderungen beschlossen, wodurch nicht mehr alle Mockups auf dem neusten Stand sind.
	\begin{itemize}
		\item Quiz-Informations-Seite\\
		Bereits umgesetzt:\\
		Auf dieser Seite wird der \glqq Start-Button\grqq nach rechts verschoben, links wird neu ein \glqq zur Übersicht\grqq - Button hinzugefügt. Zudem werden «Anzahl Fragen» und «Maximal mögliche Punktzahl» in der Reihenfolge geändert.\\
		Noch offen:\\
		Wenn der Start nicht mehr möglich ist, soll der Button ausgegraut werden.
		\item Durchführungsstatistik\\
		Noch offen:\\
		Das Datum der Durchführung soll immer angezeigt werden.
		\item Quiz-Erstellung\\
		Bereits umgesetzt:\\
		Statt \glqq Beschreibung\grqq soll \glqq Kommentar\grqq verwendet werden. Die Tabs im Quiz Erstellen sollen \glqq Allgemeine Informationen, Fragen und Durchführung\grqq heissen. Die Quiz Administration fällt weg. Das Hinzufügen von speziellen Berechtigungen wird in den Allgemeinen Informationen gemacht. Um die neuen Fragen zu erstellen wird das gleiche Konzept wie bei der Quiz Erstellung verwendet. Es wird dann vom Button \glqq Fragen erstellen\grqq auf die neue Seite \glqq Fragen erstellen\grqq weitergeleitet.
		
		\item Durchführungsoptionen\\
		Durchführungsoptionen sollen so nicht umgesetzt werden, stattdessen wird immer gleich die Einstellung in der Standardeinstellung angezeigt, wie dies bereits beim bestehenden Mobile Quiz der Fall ist. Rechts ist jeweils ein \glqq Zurücksetzen\grqq - Button, mit die jeweilige Einstellung auf den Standard-Wert zurückgesetzt werden kann.
		
	\end{itemize}
	
	\includepdf[pages={1-19}, landscape=true]{PDFs/MockupsVersion2.pdf}
	
	
	\chapter{Anpassungen an Code, Server und Konfiguration}
	%Alle Änderungen am Server, welche während der Entwicklung vorgenommen wurden
	
	\section{Anfängliche Code-Änderungen}
	\label{sec:AnfaenglicheCodeAenderungen}
	% !TEX root = Projektdokumentation.tex

\subsection{Problem: Server-Umstellung}

\textbf{Diverse Code-Stellen} \\

\noindent Ersetzen von:
\begin{lstlisting}[backgroundcolor = \color{lightgray}]
https://tlng.cnlab.ch/mobilequiz_v3/
\end{lstlisting}

\bigskip
\noindent Durch:
\begin{lstlisting}[backgroundcolor = \color{lightgray}]
http://sinv-56082.edu.hsr.ch/
\end{lstlisting}



\subsection{Problem: Keine TLS-Unterstützung durch HSR-Server}

\textbf{index.php - Zeile 5-9} \\

\noindent Auskommentieren von:
\begin{lstlisting}[backgroundcolor = \color{lightgray}]
if(empty($_SERVER['HTTPS']) || $_SERVER['HTTPS'] == "off"){
$redirect = 'https://' . $_SERVER['HTTP_HOST'] . 
				$_SERVER['REQUEST_URI'];
header('HTTP/1.1 301 Moved Permanently');
header('Location: ' . $redirect);
exit();
\end{lstlisting}



\subsection{Problem: Quiz durchzuführen}

\textbf{participation.php - Zeile 347} \\

\noindent Ersetzen von:
\begin{lstlisting}[backgroundcolor = \color{lightgray}]
$isSelected = $_POST["answer"] == $fetchAnswers[\$i]["id"];
\end{lstlisting}

\bigskip
\noindent Durch:
\begin{lstlisting}[backgroundcolor = \color{lightgray}]
if($_POST["answer"] == $fetchAnswers[$i]["id"])
{ 
$isSelected = 1; 
} 
else { 
$isSelected = 0; 
}
\end{lstlisting}



\subsection{Problem: Anzahl Teilnehmer bei Quiz-Übersicht nicht korrekt}

\textbf{in neuer Version nicht mehr vorhanden} \\

\noindent Ersetzen von:
\begin{lstlisting}[backgroundcolor = \color{lightgray}]
select id from user_qunaire_session where 
questionnaire_id = :qunaire_id group by user_id
\end{lstlisting}

\bigskip
\noindent Durch:
\begin{lstlisting}[backgroundcolor = \color{lightgray}]
select distinct user_id from user_qunaire_session 
where questionnaire_id = :qunaire_id
\end{lstlisting}



\subsection{Problem: Rangliste wird nicht angezeigt}

\textbf{Änderungen gemäss Patrick Eichler}




\subsection{Problem: Error \glqq Quiz nicht im Zeitraum\grqq bei Quiz-Ende}

\textbf{Datenbank} \\

\noindent Fehlermeldung in der Datenbank:
\begin{lstlisting}[backgroundcolor = \color{lightgray}]
sql_mode=only_full_group_by
\end{lstlisting}

\bigskip
\noindent Lösung:
\begin{lstlisting}[backgroundcolor = \color{lightgray}]
SET GLOBAL sql_mode=''; 
\end{lstlisting}



\subsection{Problem: E-Mails können nicht versendet werden}

\textbf{mail.php - Zeile 10} \\

\noindent Auskommentieren von:
\begin{lstlisting}[backgroundcolor = \color{lightgray}]
$mail->Host = '10.20.20.22';
\end{lstlisting}

\bigskip
\noindent\textbf{contact.php - Zeile 15} \\
\noindent Ersetzen von:
\begin{lstlisting}[backgroundcolor = \color{lightgray}]
mobilequiz@cnlab.ch
\end{lstlisting}

\bigskip
\noindent Durch:
\begin{lstlisting}[backgroundcolor = \color{lightgray}]
dwindler@hsr.ch
\end{lstlisting}



\subsection{Problem: Datenbank-Zugriff}

\textbf{config.php - Zeile 2-5} \\

\noindent Ersetzen von: Servername, Benutzername, Passwort und Datenbank-Name


\subsection{Problem: PDF-Generierung}

\textbf{generatePDF.php - Zeile 5} \\

\noindent Ersetzen von:
\begin{lstlisting}[backgroundcolor = \color{lightgray}]
include_once '../config/config.php';
\end{lstlisting}

\bigskip
\noindent Durch:
\begin{lstlisting}[backgroundcolor = \color{lightgray}]
include_once(__DIR__ . '/../config/config.php');
\end{lstlisting}

\bigskip
\noindent\textbf{Diverse Code-Stellen} \\

\noindent Ersetzen von:
\begin{lstlisting}[backgroundcolor = \color{lightgray}]
$stmt = $dbh->prepare("select name, description, ...
\end{lstlisting}

\bigskip
\noindent Durch:
\begin{lstlisting}[backgroundcolor = \color{lightgray}]
$stmt = $dbh->prepare("select questionnaire.name, 
description, ...
\end{lstlisting}

\noindent Grund: \glqq name\grqq war nicht eindeutig, Quelle: stackoverflow.com \cite{mysql-column-problem}


\bigskip
\noindent\textbf{Diverse Code-Stellen} \\

\noindent Vor jedem Aufruf von:
\begin{lstlisting}[backgroundcolor = \color{lightgray}]
pdf->Output
\end{lstlisting}

\bigskip
\noindent Folgender Code eingefügt:
\begin{lstlisting}[backgroundcolor = \color{lightgray}]
ob_end_clean()
\end{lstlisting}

Quelle: stackoverflow.com \cite{tcpdf-error}
	
	\section{Änderungen am Server}
	\label{sec:AenderungenAmServer}
	% !TEX root = Projektdokumentation.tex


\textbf{Wichtig:} 
Für den Server-Upload müssen die folgenden zwei Dinge gegeben sein:
\begin{itemize}
	\item Die PHP-Library 'GD' muss für die Bildverarbeitung installiert sein. Ob dies der Falls ist kann mittels PHPInfo() oder nachfolgendem Code festgestellt werden. \cite{zoopable.com} Sollte 'GD' nicht installiert sein, so ist unten ebenfalls der Installationsbefehl aufgeführt. \cite{askubuntu.com_php_extension}
\end{itemize}
\begin{lstlisting}
<?php
if (extension_loaded('gd') && function_exists('gd_info')) {
echo "PHP GD library is installed on your web server";
}
else {
echo "PHP GD library is NOT installed on your web server";
}
?>

sudo apt-get install php5.6-gd
\end{lstlisting}

\begin{itemize}
	\item Die Childprozesse des Apache-Servers, welche die Anfragen beantworten, benötigen Schreibrechte auf den Upload-Ordner. Ob dies gegeben ist, kann mittels folgendem Befehl überprüft und geändert werden. \cite{askubuntu.com_permissions} 
\end{itemize}
\begin{lstlisting}
ps -ef | grep apache | grep -v grep
\end{lstlisting}

\begin{lstlisting}
root      5001     1  0 07:21 ?    00:00:00 /usr/sbin/apache2 -k start
www-data  5021  5001  0 07:21 ?    00:00:00 /usr/sbin/apache2 -k start
www-data  5022  5001  0 07:21 ?    00:00:00 /usr/sbin/apache2 -k start
www-data  5023  5001  0 07:21 ?    00:00:00 /usr/sbin/apache2 -k start
\end{lstlisting}

Ist die Ausgabe wie oben ersichtlich, so werden die Anfragen von Prozessen der Benutzergruppe 'www-data' beantwortet. Die Schreibrechte an diese Gruppe können wie folgt vergeben werden:
\begin{lstlisting}
chgrp www-data /path/to/mydir
chmod g+w /path/to/mydir
\end{lstlisting}
	
	\section{Anleitungen}
	\label{sec:Anleitungen}
	Nachfolgend sind Anleitungen für das Aufsetzen des Servers sowie dem Einrichten eines Redmine-Backups angefügt.
	\includepdf[pages={1-4}, pagecommand={}]{PDFs/V2_Apache_PHP_MySQL-Docker.pdf}
	\includepdf[pages={1-3}, pagecommand={}]{PDFs/V3_Apache_PHP_MySQL-Ubuntu.pdf}
	\includepdf[pagecommand={}]{PDFs/Anleitung_Redmine_Datenbank_Backup.pdf}
	
	\section{Datenbankanpassungen}
	\label{sec:Datenbankanpassungen}
	Auf den folgenden Seiten sind die nötigen Datenbankbefehle aufgeführt, um den cnlab-server auf die neue Datenbankbenutzung umzustellen.
	\includepdf[pages={1-10}, pagecommand={}]{PDFs/Datenbankaenderungen.pdf}
	
	\chapter{Projektmanagementplan}
	%Projektmanagementplan (kommentierter Arbeitsplan, Zeiterfassung: Durch den Einbezug des Arbeitsplans in den Bericht sollen die Studenten zu einer bewussten Arbeitsplanung animiert werden, sodass sie lernen den Arbeitsaufwand abzuschätzen und die Arbeit optimal zu organisieren.
	
	Da für die Umsetzung der neuen Features voraussichtlich mehr Zeit benötigt wird, wird eine weitere Construction-Phase angehängt und die erste Transition-Phase gestrichen.
	
	Bemerkung zur Präsentation:
	Am Anfang wäre eine Zeitachse mit den Meilensteinen noch gut gewesen, um eine Projekt-Übersicht zu geben. Diese Zeitachse hätte man dann als roten Faden verwenden können.
	
	% !TEX root = Projektdokumentation.tex

 \section{Kostenvoranschlag}
 Das Projekt läuft im Rahmen der Studienarbeit. Diese sieht einen Personenaufwand von 240 Stunden pro Person vor, was bei einer 2-Personen-Gruppe einen Aufwand von 480 Stunden macht. 
 Der Projektrahmen ist das Herbstsemester 2016, welches vom 19.09 - 23.12.2016 dauert und somit 14 Wochen umfasst. Es ist damit ein durchschnittlicher Wochenaufwand von 17 Stunden pro Person vorgesehen.
 
 
 \section{Zeitliche Planung}
 
 \subsection{Phasen / Iterationen}
 Das Projekt ist in die Phasen Inception, Elaboration, Construction und Transition aufgeteilt. Die Inception-Phase hat bereits in der Woche vor dem Semesterbeginn stattgefunden. Die restlichen Phasen sind, wie in der Grafik auf der nächsten Seite ersichtlich, über das Herbstsemesters 2016 verteilt.
 
 \bigskip
 \textbf{Anpassung der Planung}\\
 
 Zu Beginn war die Fertigstellung des Prototyps für Woche 8 geplant. Wie bei der Besprechung der neuen Mockups festgestellt, mussten jedoch noch grundlegende Konzepte überarbeitet werden. Deshalb wurde der Meilenstein 'Prototyp' sowie der damit zusammenhängende Meilenstein 'Usability-Tests' um 1 Woche nach hinten verschoben. Die Tests werden anfangs Woche durchgeführt, wodurch noch knapp 3 Wochen Implementation verbleiben, um auf kleinere Änderungen durch die Tests zu reagieren und diese im Code umzusetzen.
 

\includepdf[landscape=true, pagecommand=\thispagestyle{mylandscape}]{PDFs/Zeitplan_Spitzenbelastung.pdf}
\includepdf[landscape=true, pagecommand=\thispagestyle{mylandscape}]{PDFs/Zeitplan_SpitzenbelastungV2.pdf}
 
 \subsubsection{Rückblick zeitliche Planung}
 Rückblickend gesehen muss gesagt werden, dass der erste grob aufgestellte Zeitplan bei weitem nicht eingehalten werden konnte. Es wurde nicht damit gerechnet, dass die Ausarbeitung der Konzepte einen so grossen Teil der Zeit in Anspruch nehmen würde. Da sich die Konzeptausarbeitungen zu Beginn der Arbeit durchgeführt wurden, verschoben sich durch den Mehraufwand alle dahinter gesetzten Termine. Auf der nächsten Seite ist der Zeitplan mit den effektiven Terminen ersichtlich. Zudem wurden auch die Belastungen der anderen Fächer gemäss dem tatsächlichen Aufwand aktualisiert.
 
\includepdf[landscape=true, pagecommand=\thispagestyle{mylandscape}]{PDFs/Zeitplan_SpitzenbelastungV3.pdf}
 
 \subsection{Meilensteine}
 %Die Meilensteine sollen konkret und messbar dargestellt werden.
 
 \begin{tabularx}{\linewidth}{|p{4.5cm}|c|c|X|}
 	\hline
 	\textbf{Meilenstein} & \textbf{Datum soll} & \textbf{Datum ist} &\textbf{Beschreibung} \\
 	\hline
 	Finalisierte Aufgabenstellung & 09.10.2016 & 09.10.2016 & Herr Heinzmann hat die zu erledigenden Arbeiten in einer Aufgabenstellung zusammengestellt und an die Studenten abgegeben. \\
 	\hline
 	End of Elaboration & 16.10.2016 & 16.10.2016 & Die Umfeldanalyse ist abgeschlossen und es ist bekannt, welche Arbeiten im Rahmen der Studienarbeit angegangen werden. \\
 	\hline
 	Erster Teil des Berichts komplett & 30.10.2016 & 25.10.2016 & Die Ergebnisse der Analyse-Phase sind vollständig niedergeschrieben, damit Herr Heinzmann diese gegenlesen kann. \\
 	\hline
 	Zwischenpräsentation & 13.11.2016 & 22.11.2016 & Die bisher erarbeiteten Ergebnisse wurden Herrn Heinzmann als Vortrag präsentiert. \\
 	\hline
 	Erster Prototyp & 13.11.2016 & 18.12.2016 & Die Code-Änderungen für eine verbesserte Usability wurden vollständig implementiert, damit in der Folgewoche die zweiten Usability-Tests durchgeführt werden können. \\
 	\hline
 	End of Construction & 11.12.2016 & 21.12.2016 & Alle Änderungen am Code wurden implementiert, sodass dieser wieder auf den cnlab-Server übertragen werden kann. \\
 	\hline
 	Schlussabgabe & 23.12.2016 & 23.12.2016 & Alle Dokumente wurden abgabekonform erstellt, die Dokumentation gebunden, der Code auf CD gebrannt und alles an Herrn Heinzmann abgegeben. \\
 	\hline
 \end{tabularx}
 
 
 
 \section{Zeiterfassung}
 Für alle Arbeiten werden in Redmine Arbeitspakete erfasst. Sofort nachdem ein Paket bearbeitet wurde, wird die Zeit darauf verbucht. Da alle Pakete einer Kategorie zugeordnet sind, kann am Ende des Projekts genau festgestellt werden, wie viel Zeit beispielsweise für alle Dokumentationen aufgewendet wurde.
 
 Die schlussendlich erfassten Pakete sind auf den folgenden Seiten abgebildet:
 
 \includepdf[landscape=true,pages={1-3}, pagecommand=\thispagestyle{mylandscape}]{PDFs/sa_mobilequiz-gantt_23-12-2016.pdf}
 
 \section{Auswertungen}
 \begin{figure}[H]
 	\centering
 	\includegraphics[width=1\textwidth]
 	{Images/ZeitauswertungSollIst.png}
 	\caption{Vergleich der Zeit Soll und Ist}
 \end{figure}
 
 Es wurden die Soll-Zeiten von 34 Stunden pro Woche den effektiv auf die Arbeitspakete gebuchten Zeiten gegenübergestellt. Die in der Mitte der Studienarbeit zu wenig geleisteten Stunden wurden gegen Ende der Studienarbeit mehr als eingeholt. Es wurden schlussendlich 502 Stunden geleistet.
 
 \bigskip
 
 Diese Stunden verteilen sich wie folgt auf die Ersteller dieser Arbeit:
 
 \begin{figure}[H]
 	\centering
 	\includegraphics[width=1\textwidth]
 	{Images/ZeitauswertungBenutzer.png}
 	\caption{Zeitauswertung der geleisteten Zeit pro Person}
 \end{figure}
 
 Im Soll / Ist Vergleich der Zeitauswertung sieht man starke Schwankungen. Wenn man allerdings die Zeitauswertung aus kumulierter Sicht betrachtet, fallen diese Schwankungen nicht mehr so stark ins Gewicht. 
 
 \begin{figure}[H]
 	\centering
 	\includegraphics[width=1\textwidth]
 	{Images/ZeitauswertungKumuliert.png}
 	\caption{Zeitauswertung aus kumulierter Sicht}
 \end{figure}

Die unten stehende Grafik zeigt die Aufwände aufgeschlüsselt auf die einzelnen Kategorien.

\begin{figure}[H]
	\centering
	\includegraphics[width=0.8\textwidth]
	{Images/ZeitauswertungKategorien.png}
	\caption{Prozentuale Verteilung der Zeit auf die Kategorien}
\end{figure}

	
	\chapter{Risikomanagement}
	%Bezug nehmen auf das Excel.
	Eine Übersicht aller technischen Risiken befindet sich auf der folgenden Seite. Darin ist der aktuelle Zustand aller uns bekannten Risiken ersichtlich.
	
	\section{Umgang mit Risiken}
	Die Teammitglieder sind bereit, bei unerwarteten oder nicht vorhergesehenen Zwischenfällen das Arbeitspensum für die Studienarbeit zu erhöhen, um den fristgerechten Abschluss der Arbeit zu gewährleisten, solange es sich dabei nicht um einen Dauerzustand handelt. Zusätzlich wird bei der Schätzung der Aufwände immer darauf geachtet, für unerwartetes eine Reserve einzuplanen.
	Nach jeder Iteration werden die bestehenden Risiken neu evaluiert und falls nötig angepasst oder neue Risiken aufgenommen.
	
	Die auf der nächsten Seite folgende Tabelle zeigt die Auswertung der Risiken, inklusive Kommentare zur Behandlung dieser Risiken, zum Ende der Studienarbeit auf.
	
	\includepdf[landscape=true]{PDFs/TechnischeRisiken.pdf} 

	\chapter{Verwendete Werkzeuge}
	\label{chap:werkzeuge}
	% !TEX root = Projektdokumentation.tex

%Verwendete Werkzeuge, Komponenten, Tools
%Bezugsquellen und Versionsnummern von externer Software


%Bezugsquellen: Es sollen alle Tools im Anhang beschrieben werden (z.B. Zotero, Codeclimate, …), inklusive Perl oder Miktex. Ziel ist, dass man mit der Beschreibung alles einrichten und direkt loslegen kann.



\section{Dokumentenverwaltung}

\begin{description}
	\item [OneDrive] ist ein bla \\
	Einsatzzweck: \\
	Version: \\
	Bezugsquelle: \\
	
	
	\item [GitHub] ist ein bla \\
	Einsatzzweck: \\
	Version: \\
	Bezugsquelle: \\
	
	
	\item [GitHubDesktop] ist ein bla \\
	Einsatzzweck: \\
	Version: \\
	Bezugsquelle: \\
\end{description}




\section{Server-Zugriff}

\begin{description}
	\item [Filezilla] ist ein bla \\
	Einsatzzweck: \\
	Version: \\
	Bezugsquelle: \\
	
	
	\item [Putty] ist ein bla \\
	Einsatzzweck: \\
	Version: \\
	Bezugsquelle: \\
\end{description}



\section{Projektverwaltung}

\begin{description}
	\item [Redmine] ist ein bla \\
	Einsatzzweck: \\
	Version: \\
	Bezugsquelle: \\
\end{description}


\section{Dokumentation}

\begin{description}
	\item [Microsoft Office (Word, Excel)] ist ein bla \\
	Einsatzzweck: \\
	Version: \\
	Bezugsquelle: \\
	
	
	\item [Latex] ist ein bla \\
	Einsatzzweck: \\
	Version: \\
	Bezugsquelle: \\
	
	
	\item [TeXstudio] ist ein bla \\
	Einsatzzweck: \\
	Version: \\
	Bezugsquelle: \\
	
	
	\item [MikTex] ist ein bla \\
	Einsatzzweck: \\
	Version: \\
	Bezugsquelle: \\
	
	
	\item [Perl] ist ein bla \\
	Einsatzzweck: \\
	Version: \\
	Bezugsquelle: \\
	
	
	\item [Zotero] ist ein bla \\
	Einsatzzweck: \\
	Version: \\
	Bezugsquelle: \\
\end{description}



\section{Software-Entwicklung}

\begin{description}
	\item [PHP Eclipse] ist ein bla \\
	Einsatzzweck: \\
	Version: \\
	Bezugsquelle: \\
	
	
	\item [Selenium Firefox Addon] ist ein bla \\
	Einsatzzweck: \\
	Version: \\
	Bezugsquelle: \\
	
	
	\item [XAMPP Control Panel] ist ein bla \\
	Einsatzzweck: \\
	Version: \\
	Bezugsquelle: \\
	
	
	\item [easy Xdebug] ist ein bla \\
	Einsatzzweck: \\
	Version: \\
	Bezugsquelle: \\
\end{description}



\section{Continuous Integration}

\begin{description}
	\item [Codeclimate] ist ein bla \\
	Einsatzzweck: \\
	Version: \\
	Bezugsquelle: \\
	
	
	\item [Travis] ist ein bla \\
	Einsatzzweck: \\
	Version: \\
	Bezugsquelle: \\
\end{description}



\section{Usability}

\begin{description}
	\item [myBalsamiq] ist ein bla \\
	Einsatzzweck: \\
	Version: \\
	Bezugsquelle: \\
\end{description}


\todo{David}
	
	
	\chapter{Kontaktadressen}
	\label{LastPage}
	%Kontaktadressen von allen beteiligten Personen (Adressen der Studenten, über welche sie auch nach der Zeit an der HSR erreicht werden können, Industriepartner, Betreuer, weitere Personen)
	\begin{multicols}{2}
	\noindent \textbf{Studenten:}
	\\
	Andrea Hauser\\
	Waldheim 1\\
	8825 Hütten\\
	E-Mail: andrea.hauser@hotmail.ch\\
	\\
	David Windler\\
	Heid 320\\
	9502 Braunau\\
	\columnbreak
	E-Mail: david.windler@windowslive.com\\
	\textbf{Betreuer:}\\
	Prof. Dr. Peter Heinzmann (Dozent)\\
	E-Mail: peter.heinzmann@hsr.ch\\
	\\
	Patrick Eichler (Assistent)\\
	E-Mail: patrick.eichler@hsr.ch\\
	
	\end{multicols}
	
	

	
	
	
\end{document}