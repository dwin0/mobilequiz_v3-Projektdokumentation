% !TEX root = Projektdokumentation.tex

Die Studienarbeit ist mein zweites grosses Informatik-Projekt im Studium und auch in meiner bisherigen Laufbahn. Da ich zuvor eine KV-Lehre absolvierte und als Quereinsteiger ins Informatik-Studium startete fehlte mir vor allem die Praxiserfahrung. Trotz noch so vielem Lernaufwand konnte diese nie weggemacht werden. Aus diesem Grund bin ich sehr froh um die drei Praktischen Projekte während des Studiums - das Engineering-Projekt, die Studienarbeit und die Bachelorarbeit. Durch diese lerne ich den theoretisch erlernten Stoff des Studiums praktisch umzusetzen und Wissenslücken selbstständig zu schliessen.

\bigskip

Zu Beginn dieser Arbeit hatte ich mit vielen der eingesetzten Technologien von Mobile Quiz noch nie gearbeitet und sehr grossen Respekt davor. Kann ich wirklich selbst einen Apache-Server aufsetzen? Wie bekomme ich die Webseite auf den Server? Alles Fragen, welche ein Informatik-Lernender leicht hätte beantworten können, ich jedoch nicht wusste.
Diese und viele weitere Fragen beschäftigten mich, bis Andrea und ich, mit grosser Unterstützung von Patrick Eichler, Apache auf einer Server-Instanz der HSR zum laufen brachten. Von da an wusste ich, dass auch die restlichen Technologien mit genügend Ehrgeiz zu erlernen sind. Dies bewahrheitete sich auch, denn das Erlernen PHP und Latex bereiteten keine Mühe und auch die SQL-Kenntnisse aus dem ersten Semester konnten schnell wieder in Erinnerung gerufen werden.

\bigskip

Während des Projektes war ich vom vielen Recherche- und Schreibaufwand überrascht und freute mich deshalb umso mehr auf das Implementieren der erarbeiteten Konzepte. Diese Phase fiel jedoch mit vielen Abgaben von anderen Modulen zusammen, wodurch sich die Stunden am Computer anhäuften. Trotzdem machte es viel Spass zu sehen, wie schnell man neue Programmiersprachen lernt, wenn man sie nur oft genug verwendet.

\bigskip

Kurz zusammengefasst: Eine intensive, aber sehr spannende und lehrreiche Zeit.