% !TEX root = Projektdokumentation.tex

\newacronym{ICTh}{ICTh}{Informatios- und Codierungstheorie}


%Einleitung mit Erkenntniss / Zusammenfassung aus allen Unterkapiteln
Beim Erstellen der neuen Mockups wurde festgestellt, dass nebst dem Aussehen auch grundsätz-liche Überlegungen zu gewissen Themen vertieft erarbeitet werden sollen. Diese Konzeptüberle-gungen umfassen die Bereiche des Gruppenmanagements, der neuen Fragetypen sowie der Statistiken und den dazugehörigen Auswertungen.
In diesem Kapitel wurde Theorie erarbeitet und dann festgelegt, wie diese Konzepte umgesetzt werden können. Beim Gruppenmanagements wurden die Rollen sowie das Arbeiten mit Gruppen analysiert und erweitert. Im Bereich der neuen Fragetypen wurde zuerst theoretisch neue Frage-Formen erarbeitet und dann die Umsetzung nach Schwierigkeit und Nützlichkeit beurteilt. Schliesslich wurde bei den Statistiken mögliche neue Auswertungsformen untersucht und aufgezeigt, welche neuen Berechnungen dazu vorgenommen werden sollen.

\section{Gruppenmanagement}
In diesem Teil wurden die Rollen sowie das bestehende Gruppenmanagement angeschaut. Zu beiden Themen wurde eine Bestandsaufnahme gemacht, welche dann um sinnvolle Konzepte und Anforderungen erweitert wurde.

\bigskip

Im Bereich Rollen wurden zusätzliche Rollen definiert. Dabei handelt es sich um den Demo-User sowie um den Assistenten.

Die Rolle des Demo-Users ist dazu gedacht, dass jemand anonym die Funktionen von MobileQuiz ausprobieren kann. Nur speziell für den Demo-User gekennzeichnete Quizzes sind dabei ersichtlich und durchführbar. Eine Quiz-Durchführungen des Demo-Users wird in den Statistiken nicht erfasst.

Mit der Rolle des Assistenten wird es für einen Ersteller möglich eine oder mehrere Personen zu bestimmen, welche ebenfalls die Quizzes des Ersteller auswerten, bearbeiten und sogar in seinem Namen erstellen können. Damit wird der Situation Assistent und Dozent aus der realen Welt Rechnung getragen.

\bigskip

Die Idee der bereits bestehenden Gruppen wurde so erweitert, dass ein Teilnehmer neu mehr als einer Gruppe zugewiesen werden kann.

Zudem kann ein Quiz neu mehrere Durchführen haben. Die Festlegung des Durchführungszeitraum sowie des Quiz-Typs wird aus dem Quiz in die Durchführung verschoben. Damit ist es möglich, das Quiz einmal zu erstellen und es dann aufgrund der gewählten Durchführungszeiträume und Quiz-Typen als unterschiedliche Durchführungen laufen zu lassen. Die Durchführung wird so konzipiert, dass ihr entweder eine Gruppe oder ein einzelner Teilnehmer hinzugefügt werden kann. 

\bigskip

Die ausführliche Ausarbeitung aller Ergebnisse ist im Anhang im Dokument \glqq Konzept Gruppenadministration\grqq ab Seite \hyperlink{page.\getpagerefnumber{pdf:gruppenadmin}}{\getpagerefnumber{pdf:gruppenadmin}} ersichtlich. Zudem finden Sie sämtliche Datenbankanpassungen, welche für diese Änderungen notwendig sind, im Dokument \glqq Datenbankmodell Konzept Gruppenadministration\grqq auf Seite \hyperlink{page.\getpagerefnumber{pdf:dbgruppenadmin}}{\getpagerefnumber{pdf:dbgruppenadmin}} sowie im Kapitel Änderungen an der Datenbank \ref{subsec:DBAenderungen}.


\section{Neue Fragetypen}
Für dieses Konzept wurde zuerst die Theorie zu unterschiedlichen Fragetypen erarbeitet. Danach wurde beurteilt, welche neuen Fragetypen wie schwierig umzusetzen sind und wie nützlich sie für MobileQuiz sind. Zudem wurde ein neues Excel-Template entwickelt, dessen detaillierte Beschreibung unter \glqq Frage-Template\grqq  \ref{subsec:FrageTemplate} ersichtlich ist.

\bigskip

\noindent Es wurden folgende Fragetypen als mögliche neue Fragetypen angeschaut:
\begin{itemize}
	\item Single-Choice und Multiple-Choice - Fragen mit Bildern als Antwort
	\item Single-Choice und Multiple-Choice - Fragen mit Bild in der Frage
	\item Freitext
	\item Lückentext
	\item Lückentext mit DropDown Auswahl
	\item Drag \& Drop
	\item Antworten Sortieren
	\item Code-Evaluation
\end{itemize}

\bigskip

Davon wurden die folgenden weiterverfolgt:
\begin{itemize}
	\item Single-Choice und Multiple-Choice - Fragen mit Bild in der Frage:
	Dieser Fragetyp wird beispielsweise bei \acrshort{CN1} verwendet, um Fragen zu einem Netzwerklayout zu stellen.
	\item Freitext:
	Obwohl dieser Fragetyp nicht implementiert wurde, so diente er doch als Inspiration für ein Feedback-Feld unter jeder Frage. Ist eine Frage für einen Studenten nicht verständlich, so kann er seine Frage über dieses Feld direkt an den Quiz-Ersteller senden.
	\item Antworten Sortieren:
	Dieser Fragetyp wäre beispielsweise bei \acrshort{CN1} nützlich, um Netzwerktechnologien nach Geschwindigkeit zu ordnen.
	\item Drag \& Drop:
	Dieser Fragetyp wäre für \acrshort{CN1} und \acrshort{ICTh} attraktiv, beispielsweise für die Bezeichnungen von Übertragungsverfahren.
\end{itemize}

\noindent Für die weiteren Fragetypen besteht derzeit kein Bedarf, weshalb auf die Umsetzung verzichtet wird.

\bigskip

Sämtliche Überlegungen zu diesem Kapitel befinden sich im Dokument \\ 'Strukturierte\_Fragetypen\_AnforderungenV2.docx', die dazu notwendigen Datenbankänderungen sind im Dokument 'newQuestions\_MobileQuizDB.mdj' festgehalten.
Das neu erarbeitete Excel-Template ist unter 'Umsetzungsvorschlag\_Frage\_TemplateV2.xlsx' ersichtlich.

\section{Statistiken und Auswertungen}
In den Statistiken und Auswertungen wurden neue Auswertungstypen angeschaut und beurteilt, welche davon umgesetzt werden sollen. Zudem wurde ausgearbeitet, wie die neuen Auswertungen zugänglich sein sollen.

\bigskip

\noindent Dabei handelt es sich um folgende neue Statistiken:
\begin{itemize}
	\item Mittelwert
	\begin{itemize}
		\item für Punkte
		\item für Zeit
	\end{itemize}
	\item Standardabweichung
	\begin{itemize}
		\item für Punkte
		\item für Zeit
	\end{itemize}
	\item Aufgabenschwierigkeit
	\item Risikobereitschaft
	\item Discrimination Index
	\item Random Guess Score
\end{itemize}

\noindent Ausser dem Random Guess Score werden alle Statistiken weiterverfolgt bzw. sollen umgesetzt werden. Der Aufwand für die tatsächliche Berechnung und Umsetzung des Random Guess Score würde den schlussendlichen Nutzen weit übersteigen.

\bigskip

Die genaue Erläuterung sowie die Berechnung finden Sie im Dokument \\ 'Strukturierte\_Statistiken\_AnforderungV2.docx'. Die dazu nötigen Datenbankänderungen sind im Dokument 'newStatistiken\_MobileQuizDB.pdf' dokumentiert.