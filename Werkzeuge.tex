% !TEX root = Projektdokumentation.tex

%Bezugsquellen: Es sollen alle Tools im Anhang beschrieben werden (z.B. Zotero, Codeclimate, …), inklusive Perl oder Miktex. Ziel ist, dass man mit der Beschreibung alles einrichten und direkt loslegen kann.


\section{Dokumentenverwaltung}

\begin{description}
	\item [OneDrive] ist ein Dienst von Microsoft, um Dateien auf einem zentralen Speicherort abzulegen. Auf diesen kann über das Internet zugegriffen werden. \cite{wikipedia_filehosting} \cite{wikipedia_oneDrive}
	\begin{itemize}
		\item Einsatzzweck: Dokumentenablage
		\item Version: 17.3
		\item Bezugsquelle: https://onedrive.live.com/about/de-de/download/
		\item Beachten: Benötigt kostenlose Registrierung auf https://onedrive.live.com/
	\end{itemize}
	
	
	\item [GitHub] ist ein Online Versionsverwaltungssystem für Software.
	\begin{itemize}
		\item Einsatzzweck: Versionskontrolle für Code und Latex-Projektdokumentation
		\item Version: unbekannt
		\item Webseite: https://github.com/
		\item Beachten: Benötigt kostenlose Registrierung auf https://github.com/join. Gratis Private-Repositories gibt es als Studenten mit der Registrierung auf https://education.github.com/pack
	\end{itemize}
	
	
	\item [GitHub Desktop] ist ein Programm für Windows und Mac OS X, um GitHub-Repositories zu lokal synchronisieren und verwalten.
	\begin{itemize}
		\item Einsatzzweck: Synchronisation von Code und Latex-Projektdokumentation
		\item Version: 3.3.1.0
		\item Bezugsquelle: https://desktop.github.com/
	\end{itemize}
\end{description}



\newpage
\section{Server-Zugriff}

\begin{description}
	\item [FileZilla Client] ist ein Programm für Windows, Mac OS X und Linux, um mittels FTP (File Transfer Protocol) Daten auf einen Server hoch- und herunterzuladen. \cite{wikipedia_filezilla}
	\begin{itemize}
		\item Einsatzzweck: Dateien auf HSR-Server hochladen
		\item Version: 3.22.1
		\item Bezugsquelle: http://filezilla.de/
	\end{itemize}
	
	
	
	\item [PuTTY] ist ein Programm für Windows und Linux, um Verbindungen mittels SSH (Secure Shell), Telnet oder über eine serielle Schnittstelle herzustellen. \cite{wikipedia_putty}
	\begin{itemize}
		\item Einsatzzweck: SSH-Verbindung zum HSR-Server, um Installationen oder Konfigurationen vorzunehmen.
		\item Version: 0.67
		\item Bezugsquelle: http://www.putty.org/
	\end{itemize}
\end{description}



\section{Projektverwaltung}

\begin{description}
	\item [Redmine] ist ein bla
	\begin{itemize}
		\item Einsatzzweck:
		\item Version:
		\item Bezugsquelle:
	\end{itemize}
\end{description}



\section{Dokumentation}

\begin{description}
	\item [Microsoft Office (Word, Excel)] ist ein bla \\
	Einsatzzweck: \\
	Version: \\
	Bezugsquelle: \\
	
	
	\item [Latex] ist ein bla \\
	Einsatzzweck: \\
	Version: \\
	Bezugsquelle: \\
	
	
	\item [TeXstudio] ist ein bla \\
	Einsatzzweck: \\
	Version: \\
	Bezugsquelle: \\
	
	
	\item [MikTex] ist ein bla \\
	Einsatzzweck: \\
	Version: \\
	Bezugsquelle: \\
	
	
	\item [Perl] ist ein bla \\
	Einsatzzweck: \\
	Version: \\
	Bezugsquelle: \\
	
	
	\item [Zotero] ist ein bla \\
	Einsatzzweck: \\
	Version: \\
	Bezugsquelle: \\
\end{description}



\section{Software-Entwicklung}

\begin{description}
	\item [PHP Eclipse] ist ein bla \\
	Einsatzzweck: \\
	Version: \\
	Bezugsquelle: \\
	
	
	\item [Selenium Firefox Addon] ist ein bla \\
	Einsatzzweck: \\
	Version: \\
	Bezugsquelle: \\
	
	
	\item [XAMPP Control Panel] ist ein bla \\
	Einsatzzweck: \\
	Version: \\
	Bezugsquelle: \\
	
	
	\item [easy Xdebug] ist ein bla \\
	Einsatzzweck: \\
	Version: \\
	Bezugsquelle: \\
\end{description}



\section{Continuous Integration}

\begin{description}
	\item [Codeclimate] ist ein bla \\
	Einsatzzweck: \\
	Version: \\
	Bezugsquelle: \\
	
	
	\item [Travis] ist ein bla \\
	Einsatzzweck: \\
	Version: \\
	Bezugsquelle: \\
\end{description}



\section{Usability}

\begin{description}
	\item [myBalsamiq] ist ein bla \\
	Einsatzzweck: \\
	Version: \\
	Bezugsquelle: \\
\end{description}


\todo{David}