% !TEX root = Projektdokumentation.tex

%Verwendete Werkzeuge, Komponenten, Tools
%Bezugsquellen und Versionsnummern von externer Software


%Bezugsquellen: Es sollen alle Tools im Anhang beschrieben werden (z.B. Zotero, Codeclimate, …), inklusive Perl oder Miktex. Ziel ist, dass man mit der Beschreibung alles einrichten und direkt loslegen kann.



\section{Dokumentenverwaltung}

\begin{description}
	\item [OneDrive] ist ein bla \\
	Einsatzzweck: \\
	Version: \\
	Bezugsquelle: \\
	
	
	\item [GitHub] ist ein bla \\
	Einsatzzweck: \\
	Version: \\
	Bezugsquelle: \\
	
	
	\item [GitHubDesktop] ist ein bla \\
	Einsatzzweck: \\
	Version: \\
	Bezugsquelle: \\
\end{description}




\section{Server-Zugriff}

\begin{description}
	\item [Filezilla] ist ein bla \\
	Einsatzzweck: \\
	Version: \\
	Bezugsquelle: \\
	
	
	\item [Putty] ist ein bla \\
	Einsatzzweck: \\
	Version: \\
	Bezugsquelle: \\
\end{description}



\section{Projektverwaltung}

\begin{description}
	\item [Redmine] ist ein bla \\
	Einsatzzweck: \\
	Version: \\
	Bezugsquelle: \\
\end{description}


\section{Dokumentation}

\begin{description}
	\item [Microsoft Office (Word, Excel)] ist ein bla \\
	Einsatzzweck: \\
	Version: \\
	Bezugsquelle: \\
	
	
	\item [Latex] ist ein bla \\
	Einsatzzweck: \\
	Version: \\
	Bezugsquelle: \\
	
	
	\item [TeXstudio] ist ein bla \\
	Einsatzzweck: \\
	Version: \\
	Bezugsquelle: \\
	
	
	\item [MikTex] ist ein bla \\
	Einsatzzweck: \\
	Version: \\
	Bezugsquelle: \\
	
	
	\item [Perl] ist ein bla \\
	Einsatzzweck: \\
	Version: \\
	Bezugsquelle: \\
	
	
	\item [Zotero] ist ein bla \\
	Einsatzzweck: \\
	Version: \\
	Bezugsquelle: \\
\end{description}



\section{Software-Entwicklung}

\begin{description}
	\item [PHP Eclipse] ist ein bla \\
	Einsatzzweck: \\
	Version: \\
	Bezugsquelle: \\
	
	
	\item [Selenium Firefox Addon] ist ein bla \\
	Einsatzzweck: \\
	Version: \\
	Bezugsquelle: \\
	
	
	\item [XAMPP Control Panel] ist ein bla \\
	Einsatzzweck: \\
	Version: \\
	Bezugsquelle: \\
	
	
	\item [easy Xdebug] ist ein bla \\
	Einsatzzweck: \\
	Version: \\
	Bezugsquelle: \\
\end{description}



\section{Continuous Integration}

\begin{description}
	\item [Codeclimate] ist ein bla \\
	Einsatzzweck: \\
	Version: \\
	Bezugsquelle: \\
	
	
	\item [Travis] ist ein bla \\
	Einsatzzweck: \\
	Version: \\
	Bezugsquelle: \\
\end{description}



\section{Usability}

\begin{description}
	\item [myBalsamiq] ist ein bla \\
	Einsatzzweck: \\
	Version: \\
	Bezugsquelle: \\
\end{description}


\todo{David}