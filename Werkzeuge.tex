% !TEX root = Projektdokumentation.tex

\section{Dokumentenverwaltung}

\begin{description}
	\item [OneDrive] ist ein Dienst von Microsoft, um Dateien auf einem zentralen Speicherort abzulegen. Auf diesen kann über das Internet zugegriffen werden. \cite{wikipedia_filehosting} \cite{wikipedia_oneDrive}
	\begin{itemize}
		\item Einsatzzweck: Dokumentenablage
		\item Version: 17.3
		\item Bezugsquelle: https://onedrive.live.com/about/de-de/download/
		\item Beachten: Benötigt kostenlose Registrierung auf https://onedrive.live.com/
	\end{itemize}
	
	
	\item [GitHub] ist ein Online Versionsverwaltungssystem für Software.
	\begin{itemize}
		\item Einsatzzweck: Versionskontrolle für Code und Latex-Projektdokumentation
		\item Version: unbekannt
		\item Webseite: https://github.com/
		\item Beachten: Benötigt kostenlose Registrierung auf https://github.com/join. Gratis Private-Repositories gibt es als Studenten mit der Registrierung auf https://education.github.com/pack
	\end{itemize}
	
	
	\item [GitHub Desktop] ist ein Programm für Windows und macOS, um GitHub-Repositories zu lokal synchronisieren und verwalten.
	\begin{itemize}
		\item Einsatzzweck: Synchronisation von Code und Latex-Projektdokumentation
		\item Version: 3.3.1.0
		\item Bezugsquelle: https://desktop.github.com/
	\end{itemize}
\end{description}



\newpage
\section{Server-Zugriff}

\begin{description}
	\item [FileZilla Client] ist ein Programm für Windows, macOS und Linux, um mittels FTP (File Transfer Protocol) Daten auf einen Server hoch- und herunterzuladen. \cite{wikipedia_filezilla}
	\begin{itemize}
		\item Einsatzzweck: Dateien auf HSR-Server hochladen
		\item Version: 3.22.1
		\item Bezugsquelle: http://filezilla.de/
	\end{itemize}
	
	
	\item [PuTTY] ist ein Programm für Windows und Linux, um Verbindungen mittels SSH (Secure Shell), Telnet oder über eine serielle Schnittstelle herzustellen. \cite{wikipedia_putty}
	\begin{itemize}
		\item Einsatzzweck: SSH-Verbindung zum HSR-Server, um Installationen oder Konfigurationen vorzunehmen.
		\item Version: 0.67
		\item Bezugsquelle: http://www.putty.org/
	\end{itemize}
\end{description}



\section{Projektverwaltung}

\begin{description}
	\item [Redmine] ist eine web-basierte Projektmanagement-Software.
	\begin{itemize}
		\item Einsatzzweck: Projektplanung, Ticketverwaltung und Zeiterfassung
		\item Version: 3.3.0.stable
		\item Bezugsquelle: Von Schule vorinstalliert.
	\end{itemize}
\end{description}


\newpage
\section{Dokumentation}

\begin{description}
	\item [Microsoft Office] ist ein Paket von Büro-Software für Windows, macOS, iOS, Android und Windows Phone. \cite{wikipedia_microsoft-office}
	\begin{itemize}
		\item Einsatzzweck: Dokumenten- und Tabellenerstellung, ausser Projektdokumentation
		\item Version: 1609
		\item Bezugsquelle: https://products.office.com/
		\item Beachten: Das Office-Paket kann als HSR-Student kostenlos heruntergeladen werden.
	\end{itemize}
	
	
	\item [TeXstudio] ist ein LaTex-Editor für Windows, macOS und Linux.
	\begin{itemize}
		\item Einsatzzweck: Erstellung von LaTex-Dokumenten, vor allem für Projektdokumentation
		\item Version: 2.11.0
		\item Bezugsquelle: http://www.texstudio.org/
		\item Beachten: Für die Erstellung von LaTex-Dokumenten benötigt es eine TeX-Distribution (siehe MiKTeX). Weiter ist ein Perl-Interpreter Voraussetzung, um ein Glossar zu erstellen.
	\end{itemize}
	
	
	\item [MiKTeX] ist eine TeX-Distribution für Windows. \cite{wikipedia_miktex}
	\begin{itemize}
		\item Einsatzzweck: Interpretation und Kompilation von LaTex-Dokumenten
		\item Version: 2.9
		\item Bezugsquelle: https://miktex.org/download
	\end{itemize}
	
	
	\item [ActivePerl] ist ein Perl-Interpreter für Windows.
	\begin{itemize}
		\item Einsatzzweck: Erstellung von LaTex-Glossaren
		\item Version: 5.24.0
		\item Bezugsquelle: http://www.activestate.com/activeperl
	\end{itemize}
	
	
	\item [Zotero] ist eine Quellenverwaltungs-Software für Windows, macOS und Linux.
	\begin{itemize}
		\item Einsatzzweck: Quellenverwaltung
		\item Version: 4.0.29.10
		\item Bezugsquelle: https://www.zotero.org/download/
		\item Beachten: Für das Speichern von neuen Quellen eignet sich das Browser-AddOn. Weiter können die Exporteinstellungen des Zotero-Standalone auf BibTeX eingestellt werden, was es ermöglicht, neue Quellen per Drag\&Drop einer .bib-Datei hinzuzufügen. So können neue Quellen schnell und einfach in LaTeX eingebunden werden.
	\end{itemize}
\end{description}



\section{Software-Entwicklung}

\begin{description}
	\item [PHP Eclipse] ist eine PHP-Entwicklungsumgebung für Windows, macOS und Linux.
	\begin{itemize}
		\item Einsatzzweck: PHP-Entwicklung
		\item Version: Neon.1 Release (4.6.1)
		\item Bezugsquelle: https://eclipse.org/pdt/
	\end{itemize}	
	
	
	\item [XAMPP Control Panel] ist eine PHP-Entwicklungsumgebung für Windows, macOS und Linux. Sie enthält Apache, MariaDB, PHP und Perl.
	\begin{itemize}
		\item Einsatzzweck: Lokale PHP-Entwicklung und Debugging. Über phpMyAdmin konnte die Datenbank leicht lokal installiert werden. Weiter ist der Apache-Server schnell eingerichtet. Da der Eclipse-Workspace im htdocs von Apache liegt, können Änderungen sofort nachvollzogen werden. Zudem ermöglicht die Kombination mit easy Xdebug (s. unten) ein einfaches Debugging mit Firefox und Eclipse.
		\item Version: 3.2.2
		\item Bezugsquelle: https://www.apachefriends.org/de/index.html
	\end{itemize}
	
	
	\item [easy Xdebug (with moveable icon)] ist ein Firefox AddOn, um einfaches Debugging mittels Eclipse zu ermöglichen.
	\begin{itemize}
		\item Einsatzzweck: Debugging mit Eclipse
		\item Version: 0.9.4
		\item Bezugsquelle: https://addons.mozilla.org/de/firefox/addon/easy-xdebug-with-moveable-
		\item Beachten: Es ist folgendermassen vorzugehen, um PHP mit Eclipse zu debuggen:
		\begin{enumerate}
			\item XAMPP Control Panel: Start von MySQL und Apache
			\item Start von Eclipse und Öffnen des Projekts
			\item Start von Firefox
			\item Firefox: Aktivierung des Toggle xdebug (roter Punkt sichtbar)
			\item Firefox: Navigation zu localhost/\textless Projektname in htdocs\textgreater
			\item Nun sollte Eclipse aufleuchten und fragen, ob in den Debug-Mode umgestellt werden soll.
		\end{enumerate}
	\end{itemize}
	
	
	\item [Selenium IDE] ist ein Firefox AddOn für Web-UI-Tests. Es ermöglicht das Aufnehmen, die Bearbeiten, das Debuggen und das Abspielen von Tests.
	\begin{itemize}
		\item Einsatzzweck: Erstellen und Abspielen von Web-UI-Tests
		\item Version: 2.9.1.1
		\item Bezugsquelle: https://addons.mozilla.org/de/firefox/addon/selenium-ide/
	\end{itemize}
\end{description}



\section{Continuous Integration}

\begin{description}
	\item [Travis] ist ein Online Continuous-Integration Service für GitHub-Projekte.
	\begin{itemize}
		\item Einsatzzweck: Builden und Unit-Testen von PHP-Code
		\item Version: unbekannt
		\item Webseite: https://travis-ci.org/ und https://travis-ci.com/
		\item Beachten: Als Student mit dem GitHub Student Developer Pack kann man unter https://travis-ci.com/ Private-Repositories kostenlos builden.
	\end{itemize}
	
	
	\item [Code Climate] ist ein Online Service, um die Code-Qualität und Test-Coverage zu messen.
	\begin{itemize}
		\item Einsatzzweck: Qualitätsmessung des PHP-Codes
		\item Version: 1.0
		\item Webseite: https://codeclimate.com/
		\item Beachten: Code Climate kann mit Travis verknüpft werden. Hat Travis alle Tests durchgeführt, so leitet er dann die Ergebnisse direkt weiter. Siehe dazu die Datei '.travis.yml'. Die Konfiguration für die Qualitäts-Tests von Code Climate sind in '.codeclimate.yml' festgelegt.
	\end{itemize}
\end{description}



\section{Usability}

\begin{description}
	\item [myBalsamiq] ist ein Online Service für die Erstellung von Mockups.
	\begin{itemize}
		\item Einsatzzweck: Erstellung von Mockups
		\item Version: Build \#release/4832 - 4832
		\item Webseite: https://www.mybalsamiq.com/
		\item Beachten: Als HSR-Student kann man dem Informatik-Studiengangleiter eine Anfrage schreiben, um kostenlos ein Projekt auf myBalsamiq erstellen zu können.
	\end{itemize}
\end{description}
