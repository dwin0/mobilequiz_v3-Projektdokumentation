% !TEX root = Projektdokumentation.tex

Mobile Quiz ist eine Online-Quiz Plattform für Computer und Smartphones, welche in verschiedenen HSR-Modulen (Computernetze, Informations- und Codierungstheorie, Informationssicherheit) und Weiterbildungskursen regelmässig eingesetzt wird. Mobile Quiz entstand 2012 aus einer Bachelorarbeit und wurde seither mehrmals erweitert. Die zu Beginn der Arbeit vorliegende Mobile Quiz Version umfasste zwar viele praktische Funktionen und Einstellungs-möglichkeiten, es mangelte aber an der Bedienbarkeit.

\bigskip

Im Rahmen dieser Studienarbeit sollten einerseits die Benutzerfreundlichkeit erhöht und andererseits neue Funktionen hinzugefügt werden.

\bigskip

Mobile Quiz wird vor allem von Dozenten verwendet, um für Ihre Studenten Quizzes in Form von Lernhilfe während des Semesters zu bieten. Es besteht weiter die Möglichkeit die Quizzes entweder mit dem Typ Testatbedingung oder als Typ Prüfung zu erstellen und durchzuspielen. Die Wichtigsten Interaktionen mit dem Mobile Quiz System wurden in der folgenden Grafik festgehalten.
\begin{figure}[H]
	\centering
	\includegraphics[width=0.75\textwidth]
	{Images/MobileQuiz_Uebersicht.PNG}
	\caption{Übersicht der wichtigsten Funktionen des Mobile Quiz}
\end{figure}

\bigskip

Das Vorgehen um sich in das Thema einzuarbeiten wurde wie folgt gewählt. In einem ersten Schritt wurde Mobile Quiz gründlich untersucht. Mit einem \gls{Usability-Test} wurden Optimierungsmöglichkeiten für Quizteilnehmer bestimmt. Anhand der Behebung von kleinen Fehlern machte man sich mit dem Code und den eingesetzten Technologien vertraut. Im Rahmen einer Umfeldanalyse wurden ähnliche Online Quiz-Plattformen gesucht, getestet und bewertet. Die aus dieser Projektphase gewonnenen Erkenntnisse halfen bei der Neugestaltung der Seiteninhalte sowie bei der Festlegung von neuen Funktionen. Beim Design wurden die angezeigten Informationen bewusst auf das Nötigste beschränkt. Zur Verbesserung der Benutzerführung wurden Symbole durch textuelle Menus ersetzt. Die Implementierung erfolgte während fünf Wochen. Da es sich bei dieser Arbeit um eine Erweiterung eines bestehenden Systems handelt, wurden die Technologien nicht neu gewählt. Es wurde die bereits bestehende Lösung mit PHP und JQuery für die Logik, der MySQL Datenbank für das Speichern der Daten und HTML zusammen mit CSS für die Darstellung verwendet. Nach der Implementierungsphase wurde die Arbeit mit einem \gls{Usability-Test} abgeschlossen.

\bigskip

Als Ergebnis dieser Arbeit konnte eine sowohl für den Quiz-Ersteller wie auch für den Teilnehmer wesentlich verbesserte Bedienbarkeit erreicht werden. Die Schritt-für-Schritt Benutzerführung erleichtert die Erstellung von Quiz, Fragen und Durchführungen. Dank der neuen Excel-Import Funktion lassen sich Quiz und Fragen einfacher erstellen. Durch die Erweiterung \glqq Fragen mit Bildern\grqq sind attraktivere Fragestellungen möglich. Die Konzeptänderung, welche pro Quiz mehrere Durchführungen möglich macht, erleichtert den Einsatz von Mobile Quiz im Unterricht mit mehreren Übungsgruppen. Dank dem neuen Design sollten sich die Quizteilnehmer schneller zurechtfinden. Dies belegt der Vergleich der Ergebnisse der beiden \gls{Usability-Test}s vor und nach der Überarbeitung des Mobile Quiz. Die jetzt vorliegende Mobile Quiz Version wird ab dem nächsten Semester produktiv eingesetzt. Die Umsetzung der in der Analysephase ausgearbeiteten statistischen Auswertungen könnte im Rahmen einer weiteren Studienarbeit erfolgen.