% !TEX root = Projektdokumentation.tex

%Die Kurzzusammenfassung (Abstract) richtet sich an Leute, die den Themenkreis der Arbeit relativ gut kennen. Für diese Leute beschreiben Sie die neuen, eigenen Resultate der Arbeit. Die Kurzzusammenfassung soll nur etwa 200 Worte (etwa 20 Zeilen) lang sein. Bei Studienarbeiten ist das von der HSR Schulleitung vorgegebene Kurzzusammenfassungsformular zu verwenden.

Mobile Quiz ist eine Online-Quiz Plattform für Computer und Smartphones, welche in verschiedenen HSR-Modulen (Computernetze, Informations- und Codierungstheorie, Informationssicherheit) und Weiterbildungskursen regelmässig eingesetzt wird. Mobile Quiz entstand 2012 aus einer Bachelorarbeit und wurde seither mehrmals erweitert. Die zu Beginn der Arbeit vorliegende Mobile Quiz Version umfasste zwar viele praktische Funktionen und Einstellungs-möglichkeiten, es mangelte aber an der Bedienbarkeit. Im Rahmen dieser Studienarbeit sollten einerseits die Benutzerfreundlichkeit erhöht und andererseits neue Funktionen hinzugefügt werden.

\bigskip

In einem ersten Schritt wurde Mobile Quiz gründlich untersucht. Mit einem Usability Test wurden Optimierungsmöglichkeiten für Quizteilnehmer bestimmt. Anhand der Behebung von kleinen Fehlern machte man sich mit dem Code und den eingesetzten Technologien vertraut. Diese umfassen PHP, HTML, CSS, Javascript, JQuery und Bootstrap. Im Rahmen einer Umfeldanalyse wurden ähnliche Online Quiz-Plattformen gesucht, getestet und bewertet. Die aus dieser Projektphase gewonnenen Erkenntnisse halfen bei der Neugestaltung der Seiteninhalte sowie bei der Festlegung von neuen Funktionen. Beim Design wurden die angezeigten Informationen bewusst auf das Nötigste beschränkt. Zur Verbesserung der Benutzerführung wurden Symbole durch textuelle Menus ersetzt. Die Implementierung erfolgte während fünf Wochen. Abgeschlossen wurde die Arbeit mit einem Usability Test.

\bigskip

Die Bedienbarkeit von Mobile Quiz wurde durch diese Studienarbeit sowohl für Quiz-Ersteller, als auch für Teilnehmer wesentlich verbessert. Die Schritt-für-Schritt Benutzerführung erleichtert die Erstellung von Quiz, Fragen und Durchführungen. Dank der neuen Excel-Import Funktion lassen sich Quiz und Fragen einfacher erstellen. Durch die Erweiterung \glqq Fragen mit Bildern\grqq sind attraktivere Fragestellungen möglich. Die Konzeptänderung, welche pro Quiz mehrere Durchführungen möglich macht, erleichtert den Einsatz von Mobile Quiz im Unterricht mit mehreren Übungsgruppen. Dank dem neuen Design sollten sich die Quizteilnehmer schneller zurechtfinden. Dies belegt der Vergleich der Ergebnisse der beiden Usability-Tests vor und nach der Überarbeitung des Mobile Quiz. Die jetzt vorliegende Mobile Quiz Version wird ab dem nächsten Semester produktiv eingesetzt. Die Umsetzung der in der Analysephase ausgearbeiteten statistischen Auswertungen könnte im Rahmen einer weiteren Studienarbeit erfolgen.