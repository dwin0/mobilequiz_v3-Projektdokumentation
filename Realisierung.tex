% !TEX root = Projektdokumentation.tex

\section{Verbesserung der bestehenden Lösung}

\subsection{Frage-Template}
Die Möglichkeit, Fragen offline zu erstellen und anschliessend hochzuladen, bestand bereits. Unterstützt wurde die Erfassung von Singechoice und Multiplechoice-Fragen. Dazu konnten Fragen in eine CSV-Datei geschrieben werden. Alle korrekten Antworten wurden mit einem Asterisk (Stern-Zeichen) versehen. Eine Multiplechoice-Frage lag vor, wenn mehr als eine Antwort mit einem Asterisk versehen war.

In dieser Arbeit wurde das CSV-Template durch ein Excel-Template abgelöst. Die Gründe dazu sind die folgenden:
\begin{itemize}
	\item Die Regeln für das Erstellen waren einem kleinen Personenkreis bekannt. Sie wurden auf der Webseite nicht beschrieben. Damit die Funktion aber Verbreitung findet, muss die Vorgehensweise öffentlich zugänglich sein.
	
	Es wurde entschieden, ein Excel-Template zum Download auf der Webseite anzubieten. Darin sind alle relevanten Informationen für die Erstellung vorhanden.
	
	\item Werden die Fragen in Excel erstellt und anschliessend daraus eine CSV-Datei generiert, so hat man die Wahl zwischen 3 verschiedenen CSV-Varianten.
	
	Der Excel-Import hingegen unterstützt alle Excel-Dateien mit der Endung .xlsx. Dieses Format ist ab Excel 2007 das Standardformat und daher weit verbreitet.
	\cite{microsoft2016}
	
	\item Das CSV-Template war auf Singlechoice und Multiplechoice - Fragen beschränkt.
	
	Da neue Fragetypen unterstützt werden sollten, wurde eine neue Struktur mittels Excel erarbeitet. Um den Ersteller zu Unterstützen wurde mit Farben und weiteren Funktionen gearbeitet, welche durch CSV nicht unterstützt werden.
\end{itemize}

Ein Beispiel des ursprünglichen Formats sowie des neuen Excel-Templates ist im Anhang, im Kapitel 19 Details zur Lösungsfindung, ersichtlich.

Der neue Template-Import wurde mit PHPExcel \cite{phpexcel} umgesetzt. Diese PHP-Library war bereits im Projekt eingebunden und wird dazu verwendet, die Rangliste aller Teilnehmenden eines Quiz zu Exportieren. Die neue Logik befindet sich hauptsächlich in der neu erstellen Datei 'importExcel.php'.

\section{Neuerungen}

\subsection{Fragen mit Bildern}
Bei jedem Fragetyp ist es neu möglich ein Bild zu hinterlegen. Die Funktion des Bild-Uploads auf den Server bestand bereits, wurde aber erweitert. So wurde zuvor eine Fehlermeldung ausgegeben, wenn ein Bild eine Grösse von über 20 Megabyte aufwies. Diese Limite wurde auf 800 Kilobyte gesenkt und falls das Bild mehr Speicher benötigt, wird es automatisch komprimiert.
\\

\textbf{Wichtig:} 
Für den Server-Upload müssen die folgenden zwei Dinge gegeben sein:
\begin{itemize}
	\item Die PHP-Library 'GD' muss für die Bildverarbeitung installiert sein. Ob dies der Falls ist kann mittels PHPInfo() oder nachfolgendem Code festgestellt werden. \cite{zoopable.com} Sollte 'GD' nicht installiert sein, so ist unten ebenfalls der Installationsbefehl aufgeführt. \cite{askubuntu.com_php_extension}
\end{itemize}
\begin{lstlisting}
<?php
if (extension_loaded('gd') && function_exists('gd_info')) {
echo "PHP GD library is installed on your web server";
}
else {
echo "PHP GD library is NOT installed on your web server";
}
?>

sudo apt-get install php5.6-gd
\end{lstlisting}

\begin{itemize}
	\item Die Childprozesse des Apache-Servers, welche die Anfragen beantworten, benötigen Schreibrechte auf den Upload-Ordner. Ob dies gegeben ist, kann mittels folgendem Befehl überprüft und geändert werden. \cite{askubuntu.com_permissions} 
\end{itemize}
\begin{lstlisting}
ps -ef | grep apache | grep -v grep
\end{lstlisting}

\begin{lstlisting}
root      5001     1  0 07:21 ?    00:00:00 /usr/sbin/apache2 -k start
www-data  5021  5001  0 07:21 ?    00:00:00 /usr/sbin/apache2 -k start
www-data  5022  5001  0 07:21 ?    00:00:00 /usr/sbin/apache2 -k start
www-data  5023  5001  0 07:21 ?    00:00:00 /usr/sbin/apache2 -k start
\end{lstlisting}

Ist die Ausgabe wie oben ersichtlich, so werden die Anfragen von Prozessen der Benutzergruppe 'www-data' beantwortet. Die Schreibrechte an diese Gruppe können wie folgt vergeben werden:
\begin{lstlisting}
chgrp www-data /path/to/mydir
chmod g+w /path/to/mydir
\end{lstlisting}












