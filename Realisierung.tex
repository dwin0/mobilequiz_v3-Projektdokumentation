% !TEX root = Projektdokumentation.tex

\section{Verbesserung der bestehenden Lösung}

\subsection{Frage-Template}
Die Möglichkeit, Fragen offline zu erstellen und anschliessend hochzuladen, bestand bereits. Unterstützt wurde die Erfassung von Singechoice und Multiplechoice-Fragen. Dazu konnten Fragen in eine CSV-Datei geschrieben werden. Alle korrekten Antworten wurden mit einem Asterisk (Stern-Zeichen) versehen. Eine Multiplechoice-Frage lag vor, wenn mehr als eine Antwort mit einem Asterisk versehen war.

In dieser Arbeit wurde das CSV-Template durch ein Excel-Template abgelöst. Die Gründe dazu sind die folgenden:
\begin{itemize}
	\item Die Regeln für das Erstellen waren einem kleinen Personenkreis bekannt. Sie wurden auf der Webseite nicht beschrieben. Damit die Funktion aber Verbreitung findet, muss die Vorgehensweise öffentlich zugänglich sein.
	
	Es wurde entschieden, ein Excel-Template zum Download auf der Webseite anzubieten. Darin sind alle relevanten Informationen für die Erstellung vorhanden.
	
	\item Werden die Fragen in Excel erstellt und anschliessend daraus eine CSV-Datei generiert, so hat man die Wahl zwischen 3 verschiedenen CSV-Varianten.
	
	Der Excel-Import hingegen unterstützt alle Excel-Dateien mit der Endung .xlsx. Dieses Format ist ab Excel 2007 das Standardformat und daher weit verbreitet.
	\cite{microsoft2016}
	
	\item Das CSV-Template war auf Singlechoice und Multiplechoice - Fragen beschränkt.
	
	Da neue Fragetypen unterstützt werden sollten, wurde eine neue Struktur mittels Excel erarbeitet. Um den Ersteller zu Unterstützen wurde mit Farben und weiteren Funktionen gearbeitet, welche durch CSV nicht unterstützt werden.
\end{itemize}

Ein Beispiel des ursprünglichen Formats sowie des neuen Excel-Templates ist im Anhang, im Kapitel 19 Details zur Lösungsfindung, ersichtlich.

Der neue Template-Import wurde mit PHPExcel \cite{phpexcel} umgesetzt. Diese PHP-Library war bereits im Projekt eingebunden und wird dazu verwendet, die Rangliste aller Teilnehmenden eines Quiz zu Exportieren. Die neue Logik befindet sich hauptsächlich in der neu erstellen Datei 'importExcel.php'.

\section{Neuerungen}






